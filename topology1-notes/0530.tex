\section*{May 30th, 2023}

Given a continuous function \(f : X \times Z \ra Y\), it induces a function \(F : Z \ra \mc{C}(X, Y)\) such that \(z \mapsto f(\cdot, z)\). Also, the converse holds (if \(X\) is locally compact Hausdorff).

\thm{46.11} Equip \(\mc{C}(X, Y)\) with the compact-open topology. If \(f : X \times Z \ra Y\) is continuous then \(F : Z \ra \mc{C}(X, Y)\) is continuous. The converse holds when \(X\) is locally compact Hausdorff.

\pf \note{\mimpd} Consider \(f\) as \(X \times Z \ra X \times \mc{C}(X, Y) \ra Y\), by \((x, z) \mapsto (x, F(z)) \mapsto F(z)(x) = f(x, z)\). Then it is a composition of \(i_X \times F\) and \(e\), which are both continuous.

\note{\mimp} Take \(z_0 \in Z\), and a subbasis element \(S(C, U)\) that contains \(F(z_0)\). We want to find a neighborhood \(W\) of \(z_0\) such that \(F(W) \subset S(C, U)\). \(F(z_0) \in S(C, U)\) is equivalent to \(F(z_0)(x) = f(x, z_0) \in U\) for all \(x \in C\), which is also equivalent to \(f(C \times z_0) \subset U\). Then by continuity, \(f\inv(U)\) is an open set in \(X \times Z\), that contains \(C \times z_0\).

Then \(f\inv(U) \cap (C \times Z)\) is an open set in \(C \times Z\), containing \(C \times z_0\). \(C\) is compact, so by the tube lemma, there exists a neighborhood \(W\) of \(z_0\) in \(Z\) such that \(C \times W \subset f\inv(U) \cap (C \times Z)\). So for all \(x \times z \in C \times W\), \(f(x, z) \in U\), which implies \(F(W) \subset S(C, U)\). \qed

\rmk For later semester: Let \(f, g\) be continuous maps \(X \ra Y\). \(f, g\) are \textbf{homotpoic} if there exists a continuous function \(h : X \times [0, 1] \ra Y\) such that \(h(x, 0) = f(x)\) and \(h(x, 1) = g(x)\) for all \(x \in X\). \(h\) is called a \textbf{homotopy} between \(f\) and \(g\).

By the above theorem, \(h\) induces a continuous function \(H : [0, 1] \ra \mc{C}(X, Y)\). In fact, a homotopy \(h\) corresponds to a path in \(\mc{C}(X, Y)\) from the point \(f\) to \(g\).

\topic{Ascoli's Theorem}

\thm. Let \(X\) be a space, \((Y, d)\) be a metric space. Equip \(\mc{C}(X, Y)\) with the compact convergence topology. Let \(\mc{F} \subset \mc{C}(X, Y)\).
\begin{enumerate}
    \item \(\mc{F}\) is equicontinuous under \(d\) and \(\mc{F}_a\) has a compact closure for every \(a \in X\).
    \item \(\mc{F}\) is contained in a compact subspace of \(\mc{C}(X, Y)\).
\end{enumerate}
(1)\mimp(2) always holds, (2)\mimp(1) if \(X\) is locally compact Hausdorff.

\pf \note{\mimp} Consider \(Y^X\) with the product topology. It is same as pointwise convergence topology, and \(Y^X\) is Hausdorff since \(Y\) is Hausdorff. Note that \(\mc{C}(X, Y)\) in the compact convergence topology is not a subspace of \(Y^X\). Let \(\mc{G} = \cl{\mc{F}}\) in \(Y^X\).

\note{Step 1} \(\mc{G}\) is a compact subspace of \(Y^X\). \\
Note that \(\mc{F} \subset \prod_{a \in X} \mc{F}_a \subset \prod_{a \in X} \cl{\mc{F}_a}\). Since \(\cl{\mc{F}_a}\) is compact, its product is also compact. (Tychonoff theorem) So \(\mc{G} = \cl{\mc{F}} \subset \prod_{a \in X} \cl{\mc{F}_a}\), and \(\mc{G}\) is a closed subset of compact set, so it is compact.

\note{Step 2} \(\mc{G}\) is equicontinuous under \(d\).\\
\(\forall x_0 \in X\), \(\forall \epsilon > 0\), there exists a neighborhood \(U\) of \(x_0\) such that for all \(f \in \mc{F}\), \(x \in U\), \(d(f(x), f(x_0)) < \frac{\epsilon}{3}\). Let \(g \in \mc{G}\) and \(x \in U\). Consider
\[
    V_x = \left\{h \in Y^X : d(h(x), g(x)) < \frac{\epsilon}{3} \text{ and } d(h(x_0), g(x_0)) < \frac{\epsilon}{3}\right\}.
\]
\(V_x\) is open in \(Y^X\), since the map \(h \mapsto h(x) \mapsto d(h(x), g(x))\) is continuous.\footnote{Why?} So \(V_x\) is a neighborhood of \(g\), so by definition of closure, \(g \in \cl{\mc{F}}\) implies that \(\exists f \in \mc{F}\) such that \(f \in V_x\). By triangle inequality,
\[
    d(g(x), g(x_0)) \leq d(g(x), f(x)) + d(f(x), f(x_0)) + d(f(x_0), g(x_0)) < \epsilon,
\]
for all \(g \in \mc{G}\) and \(x \in U\). Thus \(\mc{G}\) is equicontinuous under \(d\).

\note{Step 3} The product topology on \(Y^X\) and the compact convergence topology on \(\mc{C}(X, Y)\) coincide on \(\mc{G}\).\\
In general, the compact convergence topology is finer than the product topology, so we prove the reverse holds on \(\mc{G}\). For \(g \in \mc{G}\), take basis element \(B_C(g, \epsilon)\). We need to find an open set \(B\) containing \(g\) for the product topology in \(Y^X\) such that \(B \cap \mc{G} \subset B_C(g, \epsilon) \cap \mc{G}\).

Using the fact that \(\mc{G}\) is equicontinuous and \(C\) being compact, we can choose open sets \(\{U_{x_1}, \dots, U_{x_n}\}\) covering \(C\) such that \(d(g(x), g(x_i)) < \frac{\epsilon}{3}\) for all \(g \in \mc{G}\), \(x \in U_{x_i}\). Set
\[
    B = \left\{h \in Y^X : d(h(x_i), g(x_i)) < \frac{\epsilon}{3},\; i = 1, \dots, n\right\}.
\]
Then \(B\) is open in the product topology of \(Y^X\). Take \(h \in B \cap \mc{G}\) and \(x \in C\). Choose \(i\) so that \(x \in U_{x_i}\). From \(g, h \in \mc{G}\), \(d(h(x), h(x_i)) < \frac{\epsilon}{3}\), \(d(g(x), g(x_i)) < \frac{\epsilon}{3}\) by equicontinuity of \(\mc{G}\). By triangle inequality again,
\[
    d(h(x), g(x)) < d(h(x), h(x_i)) + d(h(x_i), g(x_i)) + d(g(x_i), g(x)) < \epsilon.
\]

\note{Step 4} \(\mc{F} \subset \mc{G} \subset \mc{C}(X, Y)\) and \(\mc{G}\) is a compact subspace of \(Y^X\), and also in \(\mc{C}(X, Y)\) with the compact convergence topology, by \sref{Step 3}. Thus \(\mc{F}\) is contained in a compact subspace \(\mc{G}\).

\note{\mimpd} Let \(\mc{H}\) be a compact subspace of \(\mc{C}(X, Y)\) containing \(\mc{F}\). We show that \(\mc{H}\) is equicontinuous and \(\mc{H}_a\) is compact for all \(a \in X\). Then \(\mc{F} \subset \mc{H}\), so \(\mc{F}\) would be equicontinuous. Also, \(\cl{\mc{F}_a} \subset \cl{\mc{H}_a} = \mc{H}_a\), so \(\cl{\mc{F}_a}\) is a closed subspace of a compact subspace, so it would be compact.

\note{Step 1} \(\mc{H}_a\) is a compact subspace of \(Y\).\\
Consider the composition \(\mc{C}(X, Y) \overset{j}{\ra} X \times \mc{C}(X, Y) \overset{e}{\ra} Y\) as \(f \mapsto a \times f \mapsto f(a)\). \(j\) is obviously continuous, and \(e\) is continuous. Since \(\mc{H}\) is compact, its continuous image \((e\circ j)(\mc{H}) = \mc{H}_a\) is compact.

\note{Step 2} \(\mc{H}\) is equicontinuous at \(a \in X\).\\
Let \(A\) be a compact subspace that contains a neighborhood \(V\) of \(a\). (\(X\) is locally compact Hausdorff) Our claim is that \(\mc{R} = \{f \mid_A : f \in \mc{H}\} \subset \mc{C}(A, Y)\) is equicontinuous at \(a\). The restriction map \(r : \mc{C}(X, Y) \ra \mc{C}(A, Y)\) is continuous, since \(r(B_C(f, \epsilon)) \subset B_C(f\mid_A, \epsilon)\). Then \(\mc{R} = r(\mc{H})\) is compact, and is a subspace of \(\mc{C}(A, Y)\) in the compact convergence topology. Since \(A\) is compact, we can consider \(\mc{C}(A, Y)\) in the uniform topology. We are now in a metric space, so compactness of \(\mc{R}\) implies that \(\mc{R}\) is totally bounded. Then totally bounded implies equicontinuity under \(d\), by \sref{Lemma 45.2}.

Then for \(\epsilon > 0\), there is a neighborhood \(U\) of \(a\) in \(A\) such that \(d(f(x), f(a)) < \epsilon\) for all \(f \in \mc{H}\), \(x \in U\). \(U \cap V\) is open in \(V\) and also in \(X\). So \(d(f(x), f(a)) < \epsilon\) for all \(x \in U \cap V\). \(\mc{H}\) is equicontinuous at \(a\). \qed

\topic{Baire Spaces}

\defn. \(A \subset X\) has empty interior if \(A\) contains no open set of \(X\) other than \(\varnothing\).

\rmk Note that \(X \bs A\) would be dense in \(X\). Any neighborhood should intersect the complement.

\defn. \note{Baire Space} \(X\) is a \textbf{Baire space} if given any countable collection \(\seq{A_n}_{n \in \N}\) of closed sets of \(X\) with empty interior in \(X\), \(\bigcup A_n\) also has empty interior in \(X\).

\ex. \(\Q\) is not a Baire space, since \(\Q\) is already a countable union. \(\N\) is a Baire space, since it is a discrete topology.

\lemma{48.1} \(X\) is a Baire space if and only if given any countable collection \(\seq{U_n}_{n \in \N}\) of open dense sets in \(X\), \(\bigcap U_n\) is also dense in \(X\).

\pf Directly follows from definition. \qed

\pagebreak
