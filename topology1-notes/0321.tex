\section*{March 21st, 2023}

We assume the axiom of choice in this lecture, so we will use it if necessary.

\rmk \note{Axiom of Choice} Given any nonempty collection \(\{A_\alpha\}\) of disjoint nonempty sets, there exists a set \(C\) consisting of exactly one element from each \(A_\alpha\).

\rmk The following are equivalent.
\begin{enumerate}
    \item Let \(\{A_\alpha\}\) be a nonempty collection of sets. If \(A_\alpha \neq \varnothing\) for all \(\alpha \in J\), then \(\prod_{\alpha \in J} A_\alpha \neq \varnothing\)
    \item The axiom of choice.
    \item If \(\{A_\alpha\}\) is a nonempty collection of nonempty sets (not necessarily disjoint) then there exists a map \(c : J \ra \bigcup_{\alpha \in J} A_\alpha\) such that \(\forall \alpha \in J\), \(c(\alpha) \in A_\alpha\). Such \(c\) is called the \textit{choice function}.
\end{enumerate}

\pf \\
\note{1\mimp2} Since \(\prod A_\alpha \neq \varnothing\), choose an element \(c \in \prod A_\alpha\). Then \(C = c(J)\).

\note{2\mimp3} \note{Lemma 9.2}. For \(\alpha \in J\), let \(A_\alpha' = \{\alpha\} \times A_\alpha\), which is not empty. Then \(\{A_\alpha'\}\) are disjoint. By axiom of choice, there exists a set \(C\) consisting of exactly one element from each \(A_\alpha'\). Therefore for each \(\alpha \in J\), there exists a unique \((\alpha, a_\alpha) \in C\) with \(a_\alpha \in A_\alpha\). Since \(C \subset \bigcup_{\alpha \in J} \paren{\{\alpha\} \times A_\alpha} \subset \bigcup_{\alpha \in J} \paren{J \times A_\alpha} = J \times \bigcup_{\alpha \in J} A_\alpha\), \(C\) is indeed a map we wanted.

\note{3\mimp1} If \(c : J \ra \bigcup_{\alpha \in J} A_\alpha\) is a choice function, itis an element of \(\prod_{\alpha \in J} A_\alpha\).

We cannot prove \sref{Theorem 19.5} without the axiom of choice!

\rmk Axiom of choice is equivalent to the following: \(\forall \alpha \in J\), let \(A_\alpha\) be a subset of some topological space \(X_\alpha\). Then
\[
    \prod_{\alpha \in J} \cl{A_\alpha} = \cl{\prod_{\alpha \in J} A_\alpha}.
\]

\pf \note{\(\supset\)} Always holds without axiom of choice. Since
\[
    \prod_{\alpha \in J} \cl{A_\alpha} = \bigcap_{\alpha \in J} \pi_\alpha\inv\paren{\cl{A_\alpha}},
\]
and \(\pi_\alpha\inv\) is continuous, \(\bigcap_{\alpha \in J} \pi_\alpha\inv\paren{\cl{A_\alpha}}\) is a closed set. But \(\prod_{\alpha \in J} A_\alpha \subset \bigcap_{\alpha \in J} \pi_\alpha\inv\paren{\cl{A_\alpha}}\), so by definition, \(\cl{\prod_{\alpha \in J} A_\alpha} \subset \bigcap_{\alpha \in J} \pi_\alpha\inv\paren{\cl{A_\alpha}}\)
because \(\cl{\prod_{\alpha \in J} A_\alpha}\) is the smallest closed set containing \(\prod_{\alpha \in J} A_\alpha\).

\note{AoC\mimp} Let \(\bf{x} = (x_\alpha) \in \prod \cl{A_\alpha}\). Then for any basis element \(U\) containing \(\bf{x}\), \(U = \prod U_\alpha\). Since \(x_\alpha \in \cl{A_\alpha}\), \(\exists y_\alpha \in U_\alpha \cap A_\alpha \neq \varnothing\). Then \(\bf{y} = (y_\alpha) \in U \cap \prod A_\alpha\).

\note{\mimp AoC} We show (1) in the above remark. Let \(X_\alpha = A_\alpha \cup \{f_\alpha\}\) (\(f_\alpha \notin A_\alpha\)), and let \(\bf{x} : J \ra \bigcup_{\alpha \in J} X_\alpha\) that maps \(\alpha \mapsto f_\alpha\). Then \(\bf{x} \in \prod_{\alpha \in J} X_\alpha\). Equip \(X_\alpha\) with indiscrete topology, then \(\cl{A_\alpha} = X_\alpha\). (We used the fact that \(A_\alpha \neq \varnothing\)). Then by assumption,
\[
    \bf{x} \in \prod_{\alpha \in J} X_\alpha = \prod \cl{A_\alpha} \subset \cl{\prod A_\alpha},
\]
So \(\prod A_\alpha \neq \varnothing\) and we are done.

\vspace*{20px}

\defn. Let \(\bf{x} = (x_1, \dots, x_n) \in \R^n\), define \(\norm{\bf{x}} = \sqrt{x_1^2 + \cdots + x_n^2}\) as a norm of \(\bf{x}\).
\begin{enumerate}
    \item The \textbf{euclidean metric} is defined as \(d(\bf{x}, \bf{y}) = \norm{\bf{x} - \bf{y}}\).
    \item The \textbf{square metric} is defined as \(\rho(\bf{x}, \bf{y}) = \max\{\abs{x_1 - y_1}, \dots, \abs{x_n - y_n}\}\).
\end{enumerate}

Check that these are indeed metrics!

\lemma{20.2} Let \(d, d'\) be metrics on \(X\), and let \(\mc{T}, \mc{T}'\) be topologies induced by \(d, d'\) respectively. Then
\begin{center}
    \(\mc{T} \subset \mc{T}' \iff \forall x \in X, \forall \epsilon > 0, \exists \delta > 0\) such that \(B_{d'}(x, \delta) \subset B_d(x, \epsilon)\).
\end{center}

\pf \\
\note{\mimp} Let \(B_d(x, \epsilon)\) be a basis element of \(\mc{T}\). Then there exists a basis element \(B'\) of \(\mc{T}'\) such that \(x \in B' \subset B_d(x, \epsilon)\). Write \(B' = B_{d'}(y, \epsilon')\) for some \(y \in X\), \(\epsilon' > 0\). Then there exists \(\delta > 0\) such that \(x \in B_{d'}(x, \delta) \subset B_{d'}(y, \epsilon')\).

\note{\mimpd} Let \(B\) be a basis element of \(\mc{T}\) containing \(x\). Then there exists \(\epsilon > 0\) such that \(B_d(x, \epsilon) \subset B\). By assumption, there exists \(B_{d'}(x, \delta) \subset B_d(x, \epsilon)\). Then \(x \in B_{d'}(x, \delta) \subset B\), so \(\mc{T}'\) is finer.

\thm{20.3} Topology on \(\R^n\) induced by \(d\), topology on \(\R^n\) induced by \(\rho\), the product topology on \(\R^n = \R\times \cdots \times \R\) are the same topology.

\pf Consider \(\bf{x} = (x_1, \dots, x_n) \in \R^n\), \(\bf{y} = (y_1, \dots, y_n) \in \R^n\). Then
\[
    \rho(\bf{x}, \bf{y}) \leq d(\bf{x}, \bf{y}) \leq \sqrt{n}\rho(\bf{x}, \bf{y}),
\]
which implies
\[
    B_\rho\paren{\bf{x}, \frac{\epsilon}{\sqrt{n}}} \subset B_d(\bf{x}, \epsilon) \subset B_\rho(\bf{x}, \epsilon).
\]
Therefore the topology induced by \(d\) and \(\rho\) are the same.

\note{\(\supset\)} Let \(B = (a_1, b_1) \times \cdots \times (a_n, b_n)\) be a basis element of the product topology on \(\R^n\), containing \(\bf{x}\). Choose \(\epsilon_i > 0\) such that \((x_i - \epsilon_i, x_i + \epsilon_i) \subset (a_i, b_i)\). Let \(\epsilon = \min_{1\leq i \leq n} \epsilon_i\). Then \(B_\rho(\bf{x}, \epsilon) \subset B\).

\note{\(\subset\)} Let \(B_\rho(\bf{x}, \epsilon)\) be a basis element of the topology induced by \(\rho\). We can write
\[
    B_\rho(\bf{x}, \epsilon) = (x_1 - \epsilon, x_1 + \epsilon) \times \cdots \times (x_n - \epsilon, x_n + \epsilon),
\]
then \(B_\rho(\bf{x}, \epsilon)\) is already a basis of the product topology on \(\R^n\).

On \(\R^\omega\), we want to define a metric, so let \(x, y \in \R^\omega\),
\[
    d(x, y) = \sqrt{\sum_{i=1}^{\infty} (x_i - y_i)^2}, \qquad \rho(x, y) = \sup_{i \in \N} \abs{x_i - y_i}.
\]
These may not exist sometimes. So use the standard bounded metric.

\defn. Let \(J\) be an index set, and \(\bf{x} = (x_\alpha), \bf{y} = (y_\alpha) \in \R^J\). Define
\[
    \bar{\rho}(\bf{x}, \bf{y}) = \sup\left\{\bar{d}(x_\alpha, y_\alpha) : \alpha \in J\right\}
\]
where \(\bar{d}\) is the standard bounded metric on \(\R\). We call this metric the \textbf{uniform metric} and this metric induces the \textbf{uniform topology} on \(\R^J\).

\thm{20.4} On \(\R^J\),
\begin{center}
    product topology \(\subsetneq\) uniform topology \(\subsetneq\) box topology.
\end{center}
The equality does not hold when \(J\) is infinite.

\pf
\note{1} Let \(\bf{x} = (x_\alpha) \in \prod U_\alpha\) where \(\prod U_\alpha\) is a basis element for the product topology. For \(\alpha_1, \dots, \alpha_n \in J\), \(U_{\alpha_i} \neq \R\) and \(U_\alpha = \R\) otherwise. For all \(i = 1, \dots, n\), take \(\epsilon_i > 0\) such that \(B_{\bar{d}}(x_{\alpha_i}, \epsilon_i) \subset U_{\alpha_i}\). (\(U_{\alpha_i}\) is open) Then \(B_{\bar{\rho}}(\bf{x}, \epsilon) \subset \prod U_\alpha\) if we take \(\epsilon = \min_{1\leq i\leq n} \epsilon_i\).

\note{2} Let \(B = B_{\bar{\rho}} (\bf{x}, \epsilon)\) be a basis of the uniform topology. Then we can write
\[
    U = \prod_{\alpha \in J} \paren{x_\alpha - \frac{\epsilon}{2}, x_\alpha + \frac{\epsilon}{2}}
\]
So we are done.

Verifying that the equality doesn't hold is left as exercise.
\pagebreak
