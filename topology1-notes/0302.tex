\setcounter{chapter}{1}
\chapter{Topological Spaces and Continuous Functions}

\section*{March 2nd, 2023}

We start from \(\R^2\). We draw two circles, and we see that they are the same. We draw another blob... are these the \textit{same}? We normally talk about the \textit{geometry}, but if we want to say that two things are the same, there should be a \textit{bijection} between two objects.

Is a circle \textit{same} with a line? Suppose that there exists a bijection between these. If I remove a point from a line, I should also remove some point from the circle. But the line has two components now, so something should be different between these two objects. What if we go to \(\R^3\) and consider knots and toruses...?

We will rigorously define this notion of \textit{same} as \textbf{homeomorphism}. We want to learn this throughout the lecture. Let's start!

You learn a lot of language and definitions in this lecture, so you might get bored, but this language is very important, if you want to learn more advanced topics.

Lets start with some set \(X\).

\topic{Topological Spaces}

\defn. \note{Topology} A \textbf{topology} on \(X\) is a collection \(\mc{T}\) of subsets of \(X\) having the following properties.
\begin{enumerate}
    \item \(\varnothing \in \mc{T}\), \(X \in \mc{T}\).
    \item The union of the elements of any subcollection \(\mc{T'}\) of \(\mc{T}\) is in \(\mc{T}\). i.e.
          \[
              \bigcup_{V \in \mc{T'}} V \in \mc{T}.
          \]
    \item The intersection of the elements of any \textbf{finite} subcollection of \(\mc{T}'\) of \(\mc{T}\) is in \(\mc{T}\). i.e.
          \[
              \bigcap_{W \in \mc{T'}} W \in \mc{T}.
          \]
\end{enumerate}
We denote the topological space \((X, \mc{T})\), and we sometimes omit the \(\mc{T}\).

\defn. \note{Open} For \(U \subseteq X\), \(U\) is an \textbf{open set} of \(X\) if \(U \in \mc{T}\).\footnote{Apply this to the above definition and check that the union, finite intersection of open sets are open.}

\ex.
\begin{enumerate}
    \item Let \(X = \{a, b, c\}\). A trivial example of topology would be \(\mc{T} = \{\varnothing, X\}\). Another example would be
          \[
              \mc{T}_1 = \{\varnothing, X, \{a\}, \{b\}, \{a, b\}\}.
          \]
          A non-example would be
          \[
              \mc{T}_2 = \{\varnothing, X, \{a, b\}, \{b, c\}\},
          \]
          we need to add \(\{b\}\) here to make it a topology.
    \item \(\mc{T} = \mc{P}(X)\) is the \textbf{discrete topology}, and \(\mc{T} = \{\varnothing, X\}\) is called the \textbf{indiscrete topology} (trivial topology).
    \item Define \(\mc{T}_f\) as
          \[
              \mc{T}_f = \{U \subseteq X : X \bs U \text{ is finite or } X\}.
          \]
          \(\mc{T}_f\) is called the \textbf{finite complement topology}.\footnote{Check that this is indeed a topology!}
\end{enumerate}

Now we want to compare topological spaces.

\defn. Let \((X, \mc{T}), (X, \mc{T'})\) be topological spaces.
\begin{enumerate}
    \item If \(\mc{T} \subset \mc{T'}\), we say that \(\mc{T'}\) is \textbf{finer} than \(\mc{T}\). If \(\mc{T}'\) \textit{properly} contains \(\mc{T}\), \(\mc{T'}\) is \textbf{strictly finer} than \(\mc{T}\). We also say that \(\mc{T}\) is \textbf{coarser} than \(\mc{T}'\), or \textbf{strictly coarser} in these two respective situations.
    \item We say that \(\mc{T}\) is \textbf{comparable} with \(\mc{T'}\) if \(\mc{T} \subset \mc{T}'\) or \(\mc{T}' \subset \mc{T}\).
\end{enumerate}

\pagebreak

\topic{Basis for a Topology}

Since we can't enumerate the entire collection \(\mc{T}\), we need a basis\footnote{This basis is different from the basis you learned in linear algebra.} to describe a topology in terms of a smaller collection of subsets of \(X\).

\defn. \note{Basis} A \textbf{basis} for a topology on \(X\) is a collection \(\mc{B}\) of subsets of \(X\) (called \textbf{basis elements}) such that
\begin{enumerate}
    \item For each \(x \in X\), there is a basis element \(B\) containing \(x\).
          \begin{center}
              \(\forall x \in X, \exists B \in \mc{B}\) s.t. \(x \in B\).
          \end{center}
    \item If \(x \in B_1 \cap B_2\), then \(\exists B_3 \in \mc{B}\) such that \(x \in B_3 \subset B_1 \cap B_2\).
\end{enumerate}

This is a way to produce a topology from the basis.

\rmk We define the \textbf{topology \(\mc{T}\) generated by \(\mc{B}\)} as follows:
\begin{center}
    A subset \(U\) of \(X\) is \textbf{open} if \(\forall x \in U\), \(\exists B\in \mc{B}\) such that \(x \in B \subset U\).
\end{center}

\pf \(\varnothing \in \mc{T}\), \(X \in \mc{T}\) is trivial by definition.

Let \(\{U_\alpha\}_{\alpha \in J}\) be an indexed family of elements of \(\mc{T}\). Then we check that \(U = \bigcup_{\alpha \in J} U_\alpha \in \mc{T}\). Let \(x \in U\), then there exists some \(\alpha\) such that \(x \in U_\alpha\). Since \(U_\alpha\) is open, \(\exists B \in \mc{B}\) such that \(x \in B \subset U_\alpha \subset U\).

Now for the finite intersection, let \(U_1, \dots, U_n \in \mc{T}\). Suppose that \(\bigcap_{i=1}^{n - 1} U_i \in \mc{T}\), to use induction. Check for the case \(n = 2\). Let \(x \in \bigcap_{i=1}^n U_i = \paren{\bigcap_{i=1}^{n - 1} U_i} \cap U_n\). We can take \(B_1, B_2 \in \mc{B}\), such that \(x \in B_1 \subset \bigcap_{i=1}^{n - 1} U_i\) and \(x \in B_2 \subset U_n\). Now we can take \(B_3 \in \mc{B}\) such that \(x \in B_3 \subset B_1 \cap B_2 \subset \bigcap_{i=1}^{n} U_i\).

\lemma{13.1} Let \(\mc{B}\) be a basis for a topology \(\mc{T}\) on \(X\). Then \(\mc{T}\) is the collection of all unions of elements of \(\mc{B}\).

\pf \note{\(\supset\)} This is clear since \(\mc{B} \subset \mc{T}\). \\
\note{\(\subset\)} Let \(U \in \mc{T}\), we want to show that \(U\) is a union of elements of \(\mc{B}\). For all \(x \in U\), \(\exists B_x \in \mc{B}\) such that \(x \in B_x \subset U\). Now note that \(\bigcup_{x \in U} B_x = U\), so \(U\) is a union of elements of \(\mc{B}\).


\pagebreak
