\chapter{Countability and Separation Axioms}

\section*{April 27th, 2023}

\topic{The Countability Axioms}

\defn. \note{Countable Basis} A space \(X\) has a \textbf{countable basis} at \(x \in X\) if there exists a collection \(\seq{U_n}\) of neighborhoods of \(x\) such that for all neighborhoods \(U\) of \(x\), there exists \(n\) such that \(U_n \subseteq U\).

\defn.
\begin{enumerate}
    \item \note{1st Countable} A space \(X\) is \textbf{1st countable} if \(X\) has a countable basis at every \(x \in X\).
    \item \note{2nd Countable} A space \(X\) is \textbf{2nd countable} if it has a countable basis.
\end{enumerate}

\ex.
\begin{enumerate}
    \item \(\R\) has a countable basis.
    \item \(\R^n\), \(\R^\omega\) with the product topology is 2nd countable.
    \item \(\R^\omega\) with the uniform topology is 1st countable but not 2nd countable.\footnote{1st countable since it is a metric space.}

          \quad \claim. If \(X\) is 2nd countable, then any discrete subspace \(A\) of \(X\) is countable.

          \quad \pf Let \(\mc{B}\) be a countable basis. For \(a \in A\), \(\exists B_a \in \mc{B}\) such that \(B_a \cap A = \{a\}\). The map \(a \mapsto B_a\) is injective, so \(A\) is countable. \qed

          \pf Consider \(A = \{\bf{x} = (x_n)_{n \in \N} : x_n = 0 \text{ or } 1, \forall n\}\).
          \(\bar{\rho}(a, b) = 1\) if \(a \neq b \in A\). Take balls centered at \(a\), with smaller radius than 1. Then it is a discrete subspace. But \(A\) has the same cardinality as \(\{0, 1\}^{\N}\), which is uncountable, contradiction. \qed
\end{enumerate}

Both countability axioms are well behaved when taking subspaces or countable products.

\thm{30.2} A subspace of 1st/2nd countable space is also 1st/2nd countable. Also, a countable product of 1st/2nd countable spaces if also 1st/2nd countable.

\pf Check by yourself! \qed

\medskip

\defn. \note{Dense} \(A \subset X\) is \textbf{dense} in \(X\) if \(\cl{A} = X\).

\thm{30.3} Let \(X\) be a 2nd countable space. Then
\begin{enumerate}
    \item \note{Lindelöf Space} Every open covering of \(X\) has a countable subcollection covering \(X\).
    \item There exists a countable subset of \(X\) that is dense in \(X\).
\end{enumerate}

\rmk The converse of above theorem holds when \(X\) is a metrizable space. (Check!)

\pf Let \(\mc{B} = \seq{B_n}_{n\in \N}\) be a countable basis.

\note{1} Let \(\mc{A}\) be an open covering of \(A\). For every \(n \in \N\), choose \(A_n \in \mc{A}\) that contains \(B_n\), if any. For index set \(J \subset \N\), let \(\mc{A}' = \seq{A_n}_{n \in J}\). We check that \(\mc{A}'\) covers \(X\). Given \(x \in X\), \(\exists A \in \mc{A}\) such that \(x \in A\). Since \(A\) is open, \(\exists B_n \in \mc{B}\) such that \(x \in B_n \subset A\). So \(n \in J\) and \(x \in B_n \subset A_n \in \mc{A}'\).

\note{2} For \(n \in \N\), if \(B_n \neq \varnothing\), choose \(x_n \in B_n\). Let \(D = \seq{x_n}_{n \in J}\). Now we show that \(\cl{D} = X\). For \(x \in X\), every \(B_n \in \mc{B}\) containing \(x\) should intersect \(D\). This is true since \(x_n \in B_n \cap D \neq \varnothing\). \qed

\ex. \(\R_l\) is 1st countable, take a collection of \(\left[x, x + \frac{1}{n}\right)\), which is a basis at \(x\). The property (2) also holds. (rationals!) Is also Lindelöf. \textit{But \(\R_l\) is not 2nd countable.}

\pf Let \(\mc{B}\) be a basis for \(\R_l\). Define a map \(x \mapsto B_x\), where \(x \in B_x \subset [x, x+1)\). This map is injective. If \(B_x = B_y\), then \(\inf B_x = \inf B_y\), so \(x = y\). \(\mc{B}\) cannot be countable. \qed

\ex. \(\R_l \times \R_l\) is not Lindelöf.\footnote{Product of Lindelöf space is not Lindelöf.}

\pagebreak

\topic{The Separation Axioms}

Recall that Hausdorff axiom was a separation axiom.

\defn. Suppose that one-points sets are closed in \(X\).
\begin{enumerate}
    \item \note{Regular} \(X\) is \textbf{regular} if for each pair of point \(x\) and a closed set \(B\) disjoint from \(x\), there exists disjoint open sets containing \(x\) and \(B\), respectively.
    \item \note{Normal} \(X\) is \textbf{normal} if for each pair of disjoint closed sets \(A, B\) of \(X\), there exists disjoint open sets containing \(A, B\), respectively.
\end{enumerate}

\rmk Normal \mimp Regular \mimp Hausdorff.

This is another way to formulate the separation axioms.

\lemma{31.1} Assume that one-point sets are closed.
\begin{enumerate}
    \item \(X\) is regular if and only if for all \(x \in X\), for any neighborhood \(U\) of \(x\), \(\exists\)neighborhood \(V\) of \(x\) such that \(\cl{V} \subset U\).
    \item \(X\) is normal if and only if for every closed set \(A\) and for any open set \(U\) containing \(A\), \(\exists\)open set \(V\) containing \(A\) such that \(\cl{V} \subset U\).
\end{enumerate}

\pf \\
\note{1} \note{\mimp} Let \(B = X \bs U\), not containing \(x\). There exists disjoint open sets \(V\) containing \(x\), and \(W\) containing \(B\). Since \(V \subset X \bs W \subset X \bs B = U\), and \(X \bs W\) is closed, \(\cl{V} \subset X \bs W\). Thus \(\cl{V} \subset U\).

\note{\mimpd} Let \(x \in X\). Given closed set \(B\) disjoint from \(x\), let \(U = X \bs B\). \(U\) is a neighborhood of \(x\), so there is a neighborhood \(V\) of \(x\) such that \(\cl{V} \subset U = X \bs B\). Then \(x \in V\) and \(B \subset X \bs \cl{V}\).

\note{2} Replace \(x\) with closed set \(A\). Check by yourself! \qed

\thm{31.2} A subspace/product of a regular space is also regular.

\pf
\note{Subspace} Let \(x \in Y\), \(x \notin B \subset Y\), \(B\) closed. We can write \(B = \cl{B} \cap Y\). Then \(x \notin \cl{B}\). Since \(X\) is regular, there exists disjoint open sets \(V, W\) with \(x \in V\), \(\cl{B} \subset W\). Take intersections with \(Y\). Then \(x \in V \cap Y\), \(B = Y \cap \cl{B} \subset Y \cap W\).

\note{Product} Let \(\{X_\alpha\}\) be a family of regular spaces. Let \(\bf{x} = (x_\alpha) \in X\), \(U\) be a neighborhood of \(\bf{x}\) in \(X\). We will use \sref{Lemma 31.1}. Choose a basis element \(\prod U_\alpha \subset U\) containing \(x\). For each \(\alpha \in J\), we can choose a neighborhood \(V_\alpha\) of \(x_\alpha\) in \(X_\alpha\) such that \(\cl{V_\alpha} \subset U_\alpha\). Here, if \(U_\alpha = X_\alpha\), choose \(V_\alpha\) as \(X_\alpha\). Then \(V = \prod V_\alpha\) is open in \(X\), and \(\cl{V} = \prod \cl{V_\alpha} \subset \prod U_\alpha \subset U\). \qed

\rmk Unfortunately, there is no analogous property for normal spaces.

\ex.
\begin{enumerate}
    \item \(\R_K\) is Hausdorff but not regular. \(0 \notin K\), and \(K\) is closed in \(\R_K\). Suppose that \(\R_K\) is regular, so that there exists disjoint open sets \(U, V\) containing \(0, K\). Choose basis element \(B\) with \(0 \in B \subset U\). Then \(B = (a, b) \bs K\), since \((a, b)\) would definitely intersect \(K\). Choose \(\frac{1}{n} \in (a, b)\), and take basis element \(C\) such that \(\frac{1}{n} \in C \subset V\). Then \(C = (c, d)\), since \(\frac{1}{n} \notin (c, d) \bs K\). Choose \(\max\left\{c, \frac{1}{n+1}\right\} < z < \frac{1}{n}\). Then \(z \in (a, b) \subset U\), \(z \in (c, d) \subset V\). \(U, V\) are not disjoint.

    \item \(\R_l\) is normal. Suppose that \(A, B\) are disjoint closed sets in \(\R_l\). Then for \(a \in A\), \(\exists [a, x_a) \subset \R_l \bs B\). (\(\R_l \bs B\) is open) Similarly for \(b\in B\), \(\exists [b, x_b) \subset \R_l \bs A\). Then \(U = \bigcup_{a \in A} [a, x_a)\), \(V = \bigcup_{b \in B} [b, x_b)\) are disjoint open sets. If not, there exists \(a \in A\), \(b\in B\) such that \([a, x_a) \cap [b, x_b) \neq \varnothing\). Since \([a, x_a) \cap B = \varnothing\), so \(b < a\). Similarly \([b, x_b) \cap A = \varnothing\), so \(a < b\). This is a contradiction.
    \item \(\R_l \times \R_l\) is not normal. (Read!)
\end{enumerate}

\topic{Normal Spaces}

Although normal spaces don't behave very well, but you will see that normal spaces are important when we have additional properties. A normal space can be metrizable if some additional property is given.

\thm{32.3} Every compact Hausdorff space is normal.

\pf Was exam question. Refer to \sref{Lemma 26.4}, \sref{Exercise 26.5}. \qed

\thm{32.2} Every metrizable space is normal.

\pf Let \(A, B\) be disjoint closed sets of a metrizable space \((X, d)\). For \(a \in A\), \(\exists \epsilon_a > 0\) such that \(B(a, \epsilon_a) \cap B = \varnothing\). (\(a \in B(a, \epsilon_a) \subset X \bs B\)) Similarly for \(b \in B\), \(\exists \epsilon_b > 0\) such that \(B(b, \epsilon_b) \cap A = \varnothing\). Take \(U = \bigcup_{a \in A} B\paren{a, \frac{\epsilon_a}{2}}\) and \(V = \bigcup_{b \in B} B\paren{b, \frac{\epsilon_b}{2}}\).\footnote{Since \(U \cap V\) may not be disjoint, we divide \(\epsilon\) by 2.} We show that \(U, V\) are disjoint. Suppose that \(z \in U \cap V\). Then \(d(a, z) < \frac{\epsilon_a}{2}\), \(d(b, z) < \frac{\epsilon_b}{2}\), so \(d(a, b) < \frac{\epsilon_a + \epsilon_b}{2}\). If \(\epsilon_a \leq \epsilon_b\), \(d(a, b) < \epsilon_b\), so \(a \in B(b, \epsilon_b)\). This contradicts \(B(b, \epsilon_b) \cap A = \varnothing\). The case \(\epsilon_b \leq \epsilon_a\) is also impossible. \qed

\pagebreak
