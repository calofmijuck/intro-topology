\section*{May 25th, 2023}

Today we will talk about function spaces.

\topic{Pointwise and Compact Convergence}

\defn. \note{Topology of Pointwise Convergence} Given \(x \in X\), an open set \(U\) of \(Y\), let
\[
    S(x, U) = \{f \in Y^X : f(x) \in U\}.
\]
The sets \(S(x, U)\) are a subbasis for topology on \(Y^X\), which is called the \textbf{topology of pointwise convergence}.

\rmk \(S(x, U) = \pi_x\inv(U)\). So it is a subbasis of product topology!

\thm{46.1} \(f_n \ra f\) in the pointwise convergence topology if and only if \(\forall x \in X\), \(f_n(x) \ra f(x)\).

\ex. \(f_n(x) = x^n \in \mc{C}(I, \R) \subset \R^I\), with \(I = [0, 1]\). Then \(f(x) = 1\) for \(x = 1\), and \(0\) otherwise. So \(\mc{C}(I, \R)\) is not closed in \(\R^I\) with the pointwise convergence topology.

\defn. \note{Topology of Compact Convergence} Let \(X\) be a topological space, \((Y, d)\) be a metric space. Define
\[
    B_C(f, \epsilon) = \left\{g \in Y^X : \sup_{x \in C} d\paren{f(x), g(x)} < \epsilon\right\}
\]
for \(f \in Y^X\), compact \(C\), \(\epsilon > 0\). The sets \(B_C(f, \epsilon)\) form a basis for topology on \(Y^X\), which is called the \textbf{topology of compact convergence}.

\rmk It is easy to check that these form a basis since \(B_C(f, \epsilon)\) is similar to a ball. If \(g \in B_C(f, \epsilon)\), then we can find \(\delta = \epsilon - \sup_{x \in C} d(f(x), g(x))\) such that \(B_C(g, \delta) \subset B_C(f, \epsilon)\). If \(g \in B_{C_1}(f_1, \epsilon_1) \cap B_{C_2}(f_2, \epsilon_2)\), we can choose \(\delta > 0\) such that \(g \in B_{C_1 \cup C_2}(g, \delta)\).

\thm{46.2} \(f_n \ra f\) in the compact convergence topology if and only if for any compact set \(C\) of \(X\), \(f_n \mid_C \ra f \mid_C\) uniformly.

\defn. \note{Compactly Generated} \(X\) is \textbf{compactly generated} if \(A\cap C\) is open/closed in \(C\) for every compact set \(C\), then \(A\) is open/closed in \(X\).

\lemma{46.3} \(X\) is compactly generated if (1) \(X\) is locally compact or (2) \(X\) is 1st countable.

\pf \note{1} Suppose that \(A \cap C\) is open in \(C\) for all compact \(C\). \(X\) is locally compact, so \(\forall x \in A\), there is a compact subspace \(C\) of \(X\) that contains a neighborhood \(U\) of \(x\). Then \(A \cap U = (A \cap C) \cap U\) is open in \(U\). Since \(U\) is open in \(X\), \(A \cap U\) is also open in \(X\). We have found a neighborhood of \(x\) contained in \(A\). \(A\) is open.

\note{2} Suppose that \(B \cap C\) is closed in \(C\) and show that \(B\) is closed. Check by yourself. \qed

\lemma{46.4} If \(X\) is compactly generated, \(f : X \ra Y\) is continuous if and only if \(f\mid_C\) is continuous for all compact subset \(C\) of \(X\).

\pf \note{\mimp} Trivial. \\
\note{\mimpd} For \(V\) open in \(Y\), \(f\inv(V) \cap C = (f\mid_C)\inv(V)\) is open. So \(f\inv(V)\) is open. \qed

\thm{46.5} Let \(X\) be a compactly generated space and \((Y, d)\) be a metric space. Then \(\mc{C}(X, Y)\) is closed in the compact convergence topology of \(Y^X\).

\pf For \(f \in \cl{\mc{C}(X, Y)}\), we show that \(f \mid_C\) is continuous for any compact \(C\). For \(n \in \N\), \(B_C\paren{f, \frac{1}{n}} \cap \mc{C}(X, Y) \neq \varnothing\). So choose sequence \(\seq{f_n}\) in \(\mc{C}(X, Y)\) converging to \(f\). Then by \sref{Lemma 46.2}, \(f_n \mid_C \ra f\mid_C\) converges uniformly and \(f\mid_C\) is continuous by the uniform limit theorem. \qed

\cor{46.6} Let \(X\) be a compactly generated space and \((Y, d)\) be a metric space. If \(\seq{f_n} \subset \mc{C}(X, Y)\) converges to \(f\) in the compact convergence topology on \(Y^X\), then \(f\) is continuous.

\thm{46.7} Let \(X\) be a space and \((Y, d)\) be a metric space. In \(Y^X\),
\begin{center}
    (pointwise convergence) \(\subseteq\) (compact convergence) \(\subseteq\) (uniform topology)
\end{center}
The first equality holds when \(X\) is discrete, since the compact sets are finite. The second equality holds when \(X\) is compact.

\rmk Note that we didn't use metric \(d\) in the pointwise convergence topology. Can we extend the other topologies without using the metric?

\defn. \note{Compact-Open Topology} Let \(S(C, U) = \left\{f \in \mc{C}(X, Y) : f(C) \subset U\right\}\) for compact \(C\) in \(X\) and open \(U\) in \(Y\). The sets \(S(C, U)\) form a subbasis for topology on \(\mc{C}(X, Y)\) that is called the \textbf{compact-open topology}.

\thm{46.8} Let \(X\) be a space and \((Y, d)\) be a metric space. On \(\mc{C}(X, Y)\), the compact-open topology and the compact convergence topology coincide.

\pf \note{\(\subset\)} Let \(f \in S(C, U)\) where \(S(C, U)\) is a subbasis for the compact-open topology. \(C\) is compact, so \(f(C)\) is compact and \(f(C) \subset U\). Define \(\Phi : f(C) \ra \R\) where \(y \mapsto d(y, Y \bs U)\). Then \(\Phi\) is continuous, also its image is always positive since \(f(C) \subset U\) and \(f(C)\) is compact. Thus there exists \(\epsilon > 0\) which is a minimum value of \(\Phi\). Then \(B_d(y, \epsilon) \cap (Y \bs U) = \varnothing\) for \(y \in f(C)\). So \(B_d(y, \epsilon) \subset U\), and it can be checked that if \(g \in B_C(f, \epsilon)\), \(d(f(x), g(x)) < \epsilon\). So \(g(x) \in B_d(f(x), \epsilon) \subset U\) and \(g(C) \subset U\). Thus \(B_C(f, \epsilon) \subset S(C, U)\).

\note{\(\supset\)} Let \(f \in B_C(f, \epsilon)\). We find a basis element, which is a finite intersection of \(S(C, U)\). i.e, \(f \in S(C_1, U_1) \cap \cdots \cap S(C_n, U_n) \subset B_C(f, \epsilon)\).

Each \(x \in X\) has a neighborhood \(V_x\) such that \(f(\cl{V_x}) \subset U_x\) and \(\diam U_x < \epsilon\). This is possible since \(f\) is continuous, \(\exists V_x\) such that \(f(V_x) \subset B_d\paren{f(x), \frac{\epsilon}{4}}\). Then \(f(\cl{V_x}) \subset B_d\paren{f(x), \frac{\epsilon}{3}}\), and the diameter is less than \(\epsilon\).

Cover \(C\) by finitely many sets \(V_{x_1}, \dots, V_{x_n}\). Then \(C_x = \cl{V_x} \cap C\) is compact since \(C\) is compact. Then \(f \in \bigcap_{i=1}^n S(C_{x_i}, U_{x_i}) \subset B_C(f, \epsilon)\). \(f\) is an element since \(f(C_{x_i}) \subset f(\cl{V_{x_i}}) \subset U_{x_i}\). The last inclusion holds since \(g(C_{x_i}), f(C_{x_i}) \subset U_{x_i}\), and \(\diam U_{x_i} < \epsilon\) and \(d(f(x), g(x)) < \epsilon\) for all \(x \in C\). (\(C\) is a cover) \qed

\cor{46.9} The compact convergence topology on \(\mc{C}(X, Y)\) does not depend on the metric of \(Y\). So if \(X\) is compact, then the uniform topology on \(\mc{C}(X, Y)\) does not depend on the metric of \(Y\).

\thm{46.10} Let \(X\) be locally compact Hausdorff. Equip \(\mc{C}(X, Y)\) with the compact-open topology. Then the evaluation map \(e : X \times \mc{C}(X, Y) \ra Y\) defined as \((x, f) \mapsto f(x)\) is continuous.

\pf Let \((x, f) \in X \times \mc{C}(X, Y)\) and \(V\) be a neighborhood of \(e(x, f) = f(x)\). \(X\) is locally compact Hausdorff, so there is a neighborhood \(U\) of \(x\) such that \(f(\cl{U}) \subset V\) and \(\cl{U}\) is compact. (\sref{Theorem 29.2} and continuity of \(f\)) Then \(U \times S(\cl{U}, V) \subset e\inv(V)\), since for any element \(x' \times f' \subset U \times S(\cl{U}, V)\), \(x' \in  U \subset \cl{U}\) so \(f'(x') \in f'(\cl{U}) \subset V\). \(U \times S(\cl{U}, V)\) is open in the compact-open topology, so \(e\inv(V)\) is open and \(e\) is continuous. \qed

\pagebreak
