\section*{March 30th, 2023}

How to construct a surjective map? We consider a partition! Each subset in the partition will be considered as an element of \(X^*\).

\(X^*\) is a \textit{partition} of \(X\) into disjoint nonempty subsets of \(X\) whose union is \(X\). Then \(p: X \ra X^*\) defined as \(p(x) = x^*\), where \(x^*\) is an element of \(X^*\) containing \(x\).

Equip \(X^*\) with the quotient topology. Then \(V\) is open in \(X^*\) if and only if \(p\inv(V)\) is open in \(X\). Meanwhile,
\[
    p\inv(V) = \bigcup_{x^* \in V} p\inv(x^*)
\]
and consider \(p\inv(x^*)\)  as a subset of \(X\).

We saw that restricting the domain of a quotient map does not necessarily give a quotient map. But in the following cases, it gives a quotient map.

\thm{22.1} Given a quotient map \(p : X \ra Y\), let \(A \subset X\) be saturated, and let \(q = p \mid_A : A \ra p(A)\).
\begin{enumerate}
    \item If \(A\) is open (or closed) in \(X\), then \(q\) is also a quotient map.
    \item If \(p\) is either an open (or closed) map then \(q\) is also a quotient map.
\end{enumerate}

\pf We prove the following claims.

\quad \claim. \(q\inv(V) = p\inv(V)\) if \(V \subset p(A)\).

\quad \pf From \(V \subset p(A)\), we have \(p\inv(V) \subset p\inv(p(A))\). We show that \(p\inv(p(A)) = A\). The left inclusion is always true. If \(x \in p\inv(p(A))\) then \(p(x) = p(a)\) for some \(a \in A\). We have \(a \in A \cap p\inv(\{p(x)\}) \neq \varnothing\). Since \(A\) is saturated, \(p\inv(\{p(x)\}) \subset A\). But \(x \in p\inv(\{p(x)\})\), so \(x \in A\). Thus \(p\inv(V) \subset A\), so we have \(q\inv(V) = p\inv(V) \cap A = p\inv(V)\).

\quad \claim. \(p(U \cap A) = p(U) \cap p(A)\) if \(U \subset X\).

\quad \pf The right inclusion is always true. Now take \(y \in p(U) \cap p(A)\), then \(\exists u \in U\), \(\exists a \in A\) such that \(y = p(u) = p(a)\). Then \(u \in p\inv(\{p(a)\}) \subset A\) since \(A\) is saturated. So \(u \in U \cap A\) and \(y = p(u) \in p(U\cap A)\).

Take \(V \subset p(A)\), assume that \(q\inv(V)\) is open in \(A\). We show that \(V\) is open in \(p(A)\).

\note{1} By assumption, \(q\inv(V) = p\inv(V)\) is open in \(X\), so \(V\) is open in \(Y\) since \(p\) is a quotient map. Then \(V = p(A) \cap V\) is open in \(p(A)\).

\note{2} Assume that \(p\) is an open map. Then \(q\inv(V) = p\inv(V) = U \cap A\) for some open set \(U\) in \(X\). Then \(V = p(p\inv(V))\) (\(p\) surjective), so \(V = p(U \cap A) = p(U) \cap p(A)\). Since \(p(U)\) is open in \(Y\), \(V\) is open in \(p(A)\).

Note that the reverse direction is trivial, since \(q = p \circ j\) is continuous.

\rmk Composition of two quotient maps is also a quotient map.

\pf Let \(f : X \ra Y\), \(g : Y \ra Z\) be quotient maps. Then for \(W \subset Z\), \((g\circ f)\inv(W)\) is open in \(X\) \miff \(g\inv(W)\) is open in \(Y\) \miff \(W\) is open in \(Z\).

\rmk If \(X\) is Hausdorff, \(X^*\) doesn't have to be Hausdorff. Take \(p: \R \ra \{a, b\}\) as \(p(x) = a\) if \(x \in \Q\), \(p(x) = b\) otherwise.

\thm{22.2} Let \(p : X \ra Y\) be a quotient map, and \(Z\) be a topological space. Let \(g : X \ra Z\) be a map that is constant on each set \(p\inv(\{y\})\) for \(y \in Y\). Then \(g\) induces a map \(f : Y \ra Z\) such that \(f \circ p = g\), and
\begin{enumerate}
    \item \(f\) is continuous if and only if \(g\) is continuous.
    \item \(f\) is a quotient map if and only if \(g\) is a quotient map.
\end{enumerate}
\[
    \begin{tikzcd}[row sep = large, column sep = large]
        X \arrow[dr, "g"] \arrow[d, "p"'] & ~ \\
        Y \arrow[r, dashed, "f"'] & Z
    \end{tikzcd}
\]

\pf Set \(f: Y \ra Z\) as \(f(y) = z\) where \(\{z\} = g(p\inv(\{y\}))\). (constant) Then for \(x \in p\inv(\{y\})\), \(\{f(p(x))\} = g(p\inv(\{p(x)\})) = \{g(x)\}\). So \(f\circ p = g\).

\note{1} \note{\mimp} Trivial, since \(g\) is a composition of two continuous maps. \\
\note{\mimpd} Let \(V\) be open in \(Z\). Then \(g\inv(V)\) is open in \(X\). \(g\inv(V) = p\inv(f\inv(V))\), so \(f\inv(V)\) is open in \(Y\) since \(p\) is a quotient map.

\note{2} \note{\mimp} Trivial, since \(g\) is a composition of two quotient maps. \\
\note{\mimpd} \(f \circ p = g\), so \(f\) is surjective.\footnote{Check this! Assume that \(g\) is surjective...} For \(V \subset Z\), assume \(f\inv(V)\) open in \(Y\). \(p\) is continuous, so \(p\inv(f\inv(V)) = g\inv(V)\) is open in \(X\). Since \(g\) is a quotient map, \(V\) is open in \(Z\).

\pagebreak

\cor{22.3} If \(g : X \ra Z\) is surjective and continuous, define
\[
    X^* = \left\{g\inv(\{z\}) : z \in Z \right\}.
\]
Equip \(X^*\) with the quotient topology. Then
\begin{enumerate}
    \item \(g\) induces a bijective continuous map \(f : X^* \ra Z\) and \(f\) is a homeomorphism if and only if \(g\) is a quotient map.
    \item If \(Z\) is Hausdorff, \(X^*\) is also Hausdorff.
\end{enumerate}
\[
    \begin{tikzcd}[row sep = large, column sep = large]
        X \arrow[dr, "g"] \arrow[d, "p"'] & ~ \\
        X^* \arrow[r, dashed, "f"'] & Z
    \end{tikzcd}
\]

\pf For \(x \in p\inv(Y)\), \(p(x) = g\inv\paren{\{g(x)\}}\). Check that \(f\) is bijective and continuous.

\note{1} \note{\mimp} Trivial. \(f\) is a homeomorphism, so it is a quotient map. \\
\note{\mimpd} \(g\) is a quotient map, so \(f\) is a quotient map. Check that for any open set \(Y\) of \(X^*\), \(V = f(Y) \subset Z\) is open. Then \(f\inv\) is continuous, so \(f\) is a homeomorphism.

\note{2} Take different points \(x^*, y^* \in X^*\). Then \(f\) is a bijection, so \(f(x^*) \neq f(y^*)\) in \(Z\). There exists disjoint neighborhoods \(U, V\) of \(f(x^*), f(y^*)\). So take inverse images then \(f\inv(U), f\inv(V)\) are disjoint neighborhoods of \(x^*, y^*\). So \(X^*\) is Hausdorff.

\ex. Let \(X = \bigcup_{n \in \N} \paren{[0, 1] \times \{n\}}\), \(Z = \left\{x \times \frac{x}{n} : x \in [0, 1],\, n \in \N\right\}\). Define \(g(x\times n) = x \times \frac{x}{n}\). \(g\) is surjective and continuous (product of continuous maps). Take \(X^* = \{g\inv(\{z\}) : z \in Z\}\) and equip \(X^*\) with the quotient topology with a quotient map \(p\). Then \(f: X^* \ra Z\) is a bijective continuous map but not a homeomorphism. We show that \(g\) is not a quotient map.

Take \(A = \{x_n\}\) where \(x_n = \frac{1}{n} \times n\), which is closed in \(X\).\footnote{Inverse of \(xy = 1\).} Then \(g(A) = \{z_n\}\) where \(z_n = \frac{1}{n} \times \frac{1}{n^2}\). \(A\) is saturated with respect to \(g\), but \(g(A)\) is not closed in \(Z\) since it has a limit point \(\bf{0}\). So \(g\) is not a quotient map.

\ex.
\begin{enumerate}
    \item Let \(X = \{x \times y : x^2 + y^2 \leq 1\} \subset \R^2\). Consider a partition \(X^*\) where all the interior points are a one-point set, and the boundary itself is another set. If we equip \(X^*\) with the quotient topology, then \(X^*\) is homeomorphic to \(S^2\).\footnote{\href{https://math.stackexchange.com/q/4274691}{Finding an explicit homeomorphism from \(D^2/S^1\) to \(S^2\)}.}
    \item Let \(X = [0, 1] \times [0, 1]\). Consider a partition \(X^*\) consisting of:
          \begin{itemize}
              \item All the interior points as a one-point set.
              \item \(\{x \times 0, x \times 1\}\) where \(0 < x < 1\).
              \item \(\{0 \times y, 1 \times y\}\) where \(0 < y < 1\).
              \item \(\{0 \times 0, 0 \times 1, 1 \times 0, 1 \times 1\}\).
          \end{itemize}
          Then \(X^*\) is homeomorphic to a torus.\footnote{\href{https://math.stackexchange.com/q/362204/329909}{Torus and Square}.}
\end{enumerate}
\pagebreak
