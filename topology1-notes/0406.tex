\section*{April 6th, 2023}

We skipped the order topology section, so we cover it now.

\defn. \note{Order Topology} Let \(X\) be a simply ordered set. Let \(\mc{B}\) be the collection of all sets of the following types.
\begin{enumerate}
    \item \((a, b) = \{x \in X : a < x < b\}\).
    \item \([a_0, b)\) where \(a_0\) is the smallest element of \(X\), if any.
    \item \((a, b_0]\) where \(b_0\) is the largest element of \(X\), if any.
\end{enumerate}
Then the topology \(\mc{T}\) generated by \(\mc{B}\) is called the \textbf{order topology}.

\rmk The open rays form a subbasis for \(\mc{T}\).
\[
    (a, \infty) = \{x : x > a\}, \quad (-\infty, a) = \{x : x < a\}.
\]

\ex. \(Y = [0, 1) \cup \{2\} \subset \R\). The singleton \(\{2\}\) is open in the subspace topology of \(Y\), but it is not open in the order topology. There isn't a basis element contained in \(\{2\}\).

\thm{16.4} Let \(X\) be an ordered set in the order topology. \(Y \subset X\) is \textbf{convex} if
\begin{center}
    \(\forall a, b \in Y,\; a < b \implies [a, b] \subset Y\).
\end{center}
If \(Y\) is convex, then the order topology on \(Y\) is equal to the subspace topology on \(Y\).

\pf Let \(R = (a, \infty) \cap Y = \{x \in Y : x > a\}\). If \(a \in Y\), then \(R\) is an open ray of \(Y\). Suppose that \(a \notin Y\). Using the convexity of \(Y\), we will show that \(a\) is either a lower bound or a upper bound of \(Y\). Suppose that there exists some element \(b, c \in Y\) such that \(b < a < c\). Then \(a \in [b, c] \subset Y\), which is a contradiction. So if \(a\) is a lower bound, \(R = Y\), or if \(a\) is an upper bound, \(R = \varnothing\). A similar argument holds for \((-\infty, a) \cap Y\).

\note{\(\supset\)} For \(a \in X\), the sets of the form \((a, \infty) \cap Y\), \((-\infty, a) \cap Y\) form a subbasis for the subspace topology on \(Y\). Since each of them is open in the order topology on \(Y\), we are done.

\note{\(\subset\)} Take open ray of \(Y\), which is a subbasis of the order topology. Open rays of \(Y\) can be written as an intersection of \(Y\) and an open ray of \(X\). So open rays of \(Y\) are open in the subspace topology on \(Y\).

\pagebreak

\thm{24.1} Let \(L\) be a linear continuum in the order topology. Then \(L\) is connected, and intervals and rays of \(L\) are connected.

\pf It is enough to show that if \(Y\) is a convex subspace of \(L\), then \(Y\) is connected. (Linear continuum is convex)

Suppose that \(Y\) is disconnected, and let \(A, B\) be a separation of \(Y\). Take \(a \in A\), \(b \in B\). Then without loss of generality, assume \(a < b\). Since \(Y\) is convex, \([a, b] \subset Y\). Now consider \(A_0 = A \cap [a, b]\), \(B_0 = B \cap [a, b]\). We can easily see that \(A_0, B_0\) is a separation of \([a, b]\) with respect to the subspace topology of \(Y\). By \sref{Theorem 16.4}, \(A_0, B_0\) is also a separation in the order topology. Take \(c = \sup A_0\), and we show that \(c \notin A_0 \cup B_0 = [a, b]\), which is a contradiction.
\begin{itemize}
    \item Suppose that \(c \in B_0\). Then \(c = b\), or \(a < c < b\). Since \(B_0\) is open, there exists an interval \((d, c] \subset B_0\). If \(c = b\), \((d, c]\) is a basis element and \(d\) is a smaller upper bound, contradicting that \(c\) is a supremum.

          If \(a < c < b\), write \((d, b] = (d, c] \cup (c, b]\). Since \(c\) is the supremum, \((c, b] \cap A_0 = \varnothing\), and \((d, c] \subset B_0\) so \((d, c] \cap A_0 = \varnothing\). Thus \((d, b] \cap A_0 = \varnothing\). This contradicts that \(c\) is a supremum, since \(x \leq d < c\) for all \(x \in A_0\), showing that \(d\) is a smaller upper bound.
    \item Suppose that \(c \in A_0\). Then \(c = a\) or \(a < c < b\). \(A_0\) is an open set, so take an interval \([c, e) \subset A_0\). There exists \(z \in L\) such that \(c < z < e\), then \(z \in A_0\). This contradicts that \(c\) is an upper bound since \(c < z \in A_0\).
\end{itemize}

\cor{24.2} \(\R\) is connected, and the intervals, rays of \(\R\) are connected.

\thm{24.3} \note{Intermediate Value Theorem} Let \(X\) be a connected set and let \(Y\) be an ordered set in the order topology. Let \(f : X \ra Y\) be a continuous function. If \(a, b \in X\) and \(r \in Y\) is a point between \(f(a)\) and \(f(b)\), then there exists \(c \in X\) such that \(f(c) = r\).

\pf Suppose that \(r \notin f(X)\). Then \(f(X) \cap (-\infty, r)\) and \(f(X) \cap (r, \infty)\) are nonempty, disjoint and open, so it is a separation of \(f(X)\). But \(f(X)\) is connected, so contradiction.

\defn. \note{Path} Let \(x, y \in X\). A \textbf{path} in \(X\) from \(x\) to \(y\) is a continuous map \(f : [a, b] \ra X\) such that \(f(a) = x\) and \(f(b) = y\).

\defn. \note{Path Connected} A space \(X\) is \textbf{path connected} if for every \(x, y \in X\), there exists a path in \(X\) from \(x\) to \(y\).

\rmk A path connected space \(X\) is connected. Otherwise \(A, B\) is a separation of \(X\). Take \(a \in A\), \(b \in B\), and consider the path \(f\) from \(a\) to \(b\). Then \(f([a, b])\) is connected, so it lies entirely in \(A\) or \(B\), then no path can join \(a \in A\) to \(b \in B\).

Path connectedness is useful because it is easier to check.

\ex.
\begin{enumerate}
    \item Take the \textbf{unit ball} \(B^n = \{\bf{x} \in \R^n : \norm{\bf{x}} \leq 1\} \subset \R^n\). \(B^n\) is path connected, and hence connected. This is because \(f(t) = (1-t)\bf{x} + t\bf{y}\) for \(0\leq t \leq 1\) is a path in \(B^n\). (Check with triangle inequality)
    \item The \textbf{punctured Euclidean space} \(\R^n \bs \{\bf{0}\}\) is path connected if \(n \geq 2\). If the line segment doesn't contain \(\bf{0}\), we are good. But if it contains \(\bf{0}\), take some broken segment.
    \item The \textbf{unit sphere} \(S^{n-1} = \{\bf{x} \in \R^n : \norm{\bf{x}} = 1\}\) is also path connected, since we can construct a surjective map from \(\R^n \bs \{\bf{0}\} \ra S^{n-1}\) by \(\bf{x} \mapsto \frac{\bf{x}}{\norm{\bf{x}}}\).\footnote{Check that a continuous image of a path connected set is also path connected!}
    \item Non-example. \(S = \left\{x \times \sin \frac{1}{x} : 0 < x < \frac{2}{\pi}\right\}\).
          Then the \textit{topologist's sine curve} \(\cl{S} = S \cup \paren{0 \times [-1, 1]}\) is connected, but not path connected.

          \pf Suppose that there exists a path \(f : [a, c] \ra \cl{S}\) such that \(f(a) = \bf{0}\) and \(f(c) = \frac{2}{\pi} \times 1\). Let \(T = \{t \in [a, c] : f(t) \in 0\times [-1, 1]\}\). Since \(T\) is an inverse image of a closed set, \(T\) has a maximum element \(b\). Consider \((f \mid_{[b, c]})(t) = \paren{x(t), y(t)}\). Then \(x(b) = 0\), and \(x(t) > 0\) for \(t > b\). Without loss of generality, consider \([b, c] = [0, 1]\). (There exists a homeomorphism) For \(n \in \N\), \(\exists u\) such that \(0 < u < x\paren{\frac{1}{n}}\) and \(\sin \frac{1}{u} = (-1)^n\). Since \(x(0) = 0\), \(\exists t_n \in \paren{0, \frac{1}{n}}\) such that \(x(t_n) = u\), by \sref{IVT}. Then \(y(t_n) = \sin \frac{1}{x(t_n)} = (-1)^n\), so \(y(t_n)\) does not converge while \(t_n\) converges to \(0\). This is a contradiction, since \(y(t)\) is a continuous function.
\end{enumerate}

\topic{Components and Local Connectedness}

\defn. \note{Component} Let \(X\) be a space. Then we give a equivalence relation on \(X\) by
\begin{center}
    \(x \sim y \iff \exists\) a connected subspace of \(X\) containing \(x\) and \(y\).
\end{center}
The equivalences classes of \(X\) are called the (connected) \textbf{components}.

\rmk This is indeed an equivalence relation. For transitivity, use the common point theorem.

\pagebreak
