\section*{March 7th, 2023}

\ex. Examples of Bases
\begin{enumerate}
    \item Let \(\mc{B}\) be the collection of all circular regions in \(\R^2\), and let \(\mc{B}'\) be the collection of all rectangular regions in \(\R^2\), with sides parallel to the axis.

          Check that these two are indeed bases, and these two generate the same topology in \(\R^2\).

    \item The collection of all one-point subsets of \(X\) is a basis for the discrete topology on \(X\).
\end{enumerate}

Now let's try to construct a basis from the topology.

\lemma{13.2} Let \((X, \mc{T})\) be a topological space and let \(\mc{C} \subset \mc{T}\). If
\begin{center}
    \(\forall U \in \mc{T}\), \(\forall x \in U\), \(\exists C \in \mc{C}\) such that \(x \in C \subset U\).
\end{center}
Then \(\mc{C}\) is a basis for \(\mc{T}\).

\pf We need to show that \(\mc{C}\) is a basis and that \(\mc{C}\) generates \(\mc{T}\).

\note{Basis} \(\forall x \in X\), \(\exists C \in \mc{C}\) such that \(x \in C \subset X\). If \(x \in C_1 \cap C_2\), since \(C_1 \cap C_2\) is open, \(\exists C_3 \in \mc{C}\) such that \(x \in C_3 \subset C_1\cap C_2\). (\(\mc{C}\) is a subcollection of \(\mc{T}\))

\note{Generates \(\mc{T}\)} Let \(\mc{T}'\) be the topology generated by \(\mc{C}\). If \(U \in \mc{T}\), then \(\forall x \in U\), \(\exists C \in \mc{C}\) such that \(x \in C \subset U\). So \(U \in \mc{T}'\) by definition. If \(W \in \mc{T}'\), then \(W\) is a union of elements of \(\mc{C}\). Since \(\mc{C} \subset \mc{T}\) and \(\mc{T}\) is a topology, \(W \in \mc{T}\).

The following lemma shows that we can compare topologies using the bases.

\lemma{13.3} Let \(\mc{B}, \mc{B}'\) be bases of topologies \(\mc{T}, \mc{T}'\) on \(X\), respectively. TFAE.
\begin{enumerate}
    \item \(\mc{T} \subset \mc{T'}\) (\(\mc{T}'\) is finer)
    \item \(\forall x \in X\), \(\forall B \in \mc{B}\), \(\exists B'\in \mc{B}\) such that \(x \in B' \subset B\).
\end{enumerate}

\pf Left as exercise.

Now we can see that the bases in the above {\sffamily Example} (1) generate the same topology. So different bases can generate the same topology.

\ex. For \(X = \R\),
\begin{enumerate}
    \item Let \(\mc{B}\) be the collection of all open intervals \((a, b) = \{x \in \R : a < x < b\}\). Then the topology \(\mc{T}\) generated by \(\mc{B}\) is called the \textbf{standard topology on \(\R\)}.
    \item Let \(\mc{B}'\) be the collection of all half-open intervals \([a, b) = \{x \in \R : a \leq x < b\}\). Then the topology \(\mc{T}'\) generated by \(\mc{B}'\) is called the \textbf{lower limit topology on \(\R\)} and write \(\R_l\).
    \item Let \(\mc{B}''\) be the collection of all open intervals \((a, b)\) and all sets of the form \((a, b) \bs K\) where \(K = \left\{\frac{1}{n} : n \in \N\right\}\). Then the topology \(\mc{T}''\) generated by \(\mc{B''}\) is called the \textbf{\(K\)-topology} on \(\R\) and write \(\R_K\).
\end{enumerate}

Check that the above examples are indeed a basis! With these bases, we compare these topologies.

\lemma{13.4}
\begin{enumerate}
    \item \(\mc{T} \subsetneq \mc{T''}\) and \(\mc{T} \subsetneq \mc{T'}\).
    \item \(\mc{T'}\) and \(\mc{T''}\) are not comparable.
\end{enumerate}

\pf Left as exercise.

\defn. \note{Subbasis} A \textbf{subbasis} \(\mc{S}\) for a topology on \(X\) is a collection of subsets of \(X\) whose union equals \(X\).\footnote{Only requires the first condition of bases.} The \textbf{topology \(\mc{T}\) generated by the subbasis \(\mc{S}\)} is the collection of all unions of finite intersections of elements of \(\mc{S}\).

The collection \(\mc{B}\) of all finite intersections of elements of \(\mc{S}\) is a basis, since intersection of two finite intersection is also finite.

\ex. The open rays \((a, \infty)\) or \((-\infty, b)\) form a subbasis of \(\R\).

\topic{The Order Topology}

Read by yourself!

\topic{The Product Topology on \(X \times Y\)}

Let \(X\), \(Y\) be topological spaces.

\defn. Let
\[
    \mc{B} = \{U \times V: U \text{ is open in } X, V \text{ is open in } Y\}.
\]
Then the topology \(\mc{T}\) generated by \(\mc{B}\) is called the \textbf{product topology on \(X \times Y\)}.

\thm{15.1} \note{Product Topology} Let \(\mc{B}\), \(\mc{C}\) be bases of \(\mc{T}_X\), \(\mc{T}_Y\) respectively. Then
\[
    \mc{D} = \{B\times C : B\in \mc{B}, C \in \mc{C}\}
\]
is a basis for the product topology of \(X \times Y\).

\pf Left as exercise.

\ex. The product topology on \(\R^2 = \R \times \R\) is called the \textbf{standard topology} on \(\R^2\).

\defn. Define projections \(\pi_1 : X\times Y \ra X\), \(\pi_2 : X\times Y \ra Y\) as
\[
    \pi_1(x, y) = x, \quad \pi_2(x, y) = y, \quad (x \in X, y\in Y).
\]

\thm{15.2}
\[
    \mc{S} = \{\pi_1\inv(U): U \text{ is open in } X\} \cup \{\pi_2\inv(V) : V \text { is open in } Y\}
\]
is a subbasis for the product topology on \(X\times Y\).

\pf Check that \(\pi_1\inv(U) \cap \pi_2\inv(V) = U \times V\).

\topic{The Subspace Topology}

We want to give a topology on the subset of \(X\), with the topology of \(X\).

\defn. For \(Y \subset X\),
\[
    \mc{T}_Y = \{Y \cap U : U \in \mc{T}\}
\]
is called the \textbf{subspace topology on \(Y\)}.

The natural way to construct a basis on the subspace topology would be to take intersection of the original basis. Check {\sffamily Lemma 16.1}.

\rmk Let \(X = \R\) and \(Y = [0, 1]\). Note that \([0, 1)\) is not open in \(X\) but open in \(Y\).

\thm{16.3} Let \(A, B\) be subspaces of \(X, Y\) respectively. Then the product topology on \(A \times B\) is equal to the subspace topology of \(A \times B \subset X \times Y\).

\pf Let \(U \times V\) be a basis element for \(X \times Y\). Note that \((U\times V) \cap (A \times B) = (U\cap A) \times (V \cap B)\). The left hand side is a basis element of the subspace topology of \(A \times B \subset X \times Y\) and the right hand side is a basis of the product topology on \(A \times B\).

\topic{Closed Sets and Limit Points}

\defn. \note{Closed} Let \(A \subset X\). \(A\) is \textbf{closed} if \(X \bs A\) is open.

\thm{17.1} Let \(X\) be a topological space. Then the following hold.
\begin{enumerate}
    \item \(\varnothing, X\) are closed.
    \item Arbitrary intersection of closed sets are closed.
    \item Finite union of closed sets are closed.
\end{enumerate}

\pf Left as exercise.

\thm{17.2} Let \(Y\) be a subspace of \(X\). For \(A \subset Y\), \(A\) is closed if and only if there exists a closed set \(C\) of \(X\) such that \(A = C \cap Y\).

\defn. \note{Interior \& Closure} Let \(A \subseteq X\).
\begin{enumerate}
    \item The interior of \(A\), denoted as \(\inte{A}\), is the union of all open sets \(U \subseteq A\).
    \item The closure of \(A\), denoted as \(\cl{A}\), is the intersection of all closed sets \(A \subseteq C\).
\end{enumerate}

Review some facts from analysis!

\thm{17.4} Let \(Y\) be a subspace of \(X\), and \(A \subseteq Y\). Then the closure of \(A\) in \(Y\) is equal to \(\cl{A} \cap Y\).

\pf Let \(B\) be the closure of \(A\) in \(Y\).

\note{\(\subset\)} Since \(\cl{A}\) is closed in \(X\), \(\cl{A} \cap Y\) is closed in \(Y\). Since \(B\) is the smallest closed set containing \(A\), this inclusion is true.

\note{\(\supset\)} There exists a closed set \(C\) of \(X\) such that \(B = C \cap Y\). Then \(\cl{A} \subset C\), so we are done.

\thm{17.5} Let \(A \subseteq X\), with basis \(\mc{B}\).
\begin{enumerate}
    \item \(x \in \cl{A} \iff\) For any open set \(U\) containing \(x\), \(U \cap A \neq \varnothing\).
          \item\(x \in \cl{A} \iff\) For all \(B \in \mc{B}\) containing \(x\), \(B \cap A \neq \varnothing\).
\end{enumerate}

\pf \note{1} Prove the contrapositive. \(x \notin \cl{A} \iff\) there exists an open set \(U\) containing \(x\) such that \(U \cap A = \varnothing\).
\pagebreak
