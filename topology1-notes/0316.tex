\section*{March 16th, 2023}

This is a theorem about the basis of the product topology and the box topology.

\thm{19.2} Let \(\mc{B}_\alpha\) be a basis for the topology on \(X_\alpha\). Then,
\[
    \left\{\prod_{\alpha \in J} B_\alpha : B_\alpha \in \mc{B}_\alpha \text{ for finitely many } \alpha,\, B_\alpha = X_\alpha \text{ otherwise}\right\}
\]
is a basis for the product topology on \(\prod X_\alpha\). Also,
\[
    \left\{\prod_{\alpha \in J} B_\alpha : B_\alpha \in \mc{B}_\alpha\right\}
\]
is a basis for the box topology.

\pf Left as exercise.

The following theorem is a generalization of \sref{Theorem 16.3}.

\thm{19.3} Let \(A_\alpha\) be a subspace of \(X_\alpha\). Then \(\prod_{\alpha \in J} A_\alpha\) is a subspace of \(\prod_{\alpha \in J} X_\alpha\) if both products are given by the product/box topology.

\thm{19.4} If \(X_\alpha\) is Hausdorff, then \(\prod_{\alpha \in J}\) is also Hausdorff, for both product and box topology.

\thm{19.5} Let \(A_\alpha \subseteq X_\alpha\). Then
\[
    \prod_{\alpha \in J} \cl{A_\alpha} = \cl{\prod_{\alpha \in J} A_\alpha}
\]
for both product and box topology.

\pf \note{\(\subseteq\)} Take \(\bf{x} = (x_\alpha) \in \prod \cl{A_\alpha}\). Let \(U\) be a basis element containing \(\bf{x}\). Then \(U = \prod U_\alpha\), and since \(x_\alpha \in \cl{A_\alpha}\), \(U_\alpha \cap A_\alpha \neq \varnothing\) for all \(\alpha \in J\). Choose \(y_\alpha \in U_\alpha \cap A_\alpha\), then \(\bf{y} = (y_\alpha) \in U\cap \prod A_\alpha\), so \(\bf{y} \in \cl{\prod A_\alpha}\).

\note{\(\supseteq\)} Let \(\bf{x} \in \cl{\prod A_\alpha}\), we show that \(x_\beta \in \cl{A_\beta}\). Let \(V_\beta\) be a neighborhood of \(x_\beta\) in \(X_\beta\). Then \(\bf{x} \in \pi_{\beta}\inv(V_\beta)\) is open in \(\prod X_\alpha\). So there exists \(\bf{y} = (y_\alpha) \in \pi_\beta\inv(V_\beta) \cap \prod A_\alpha \neq \varnothing\). Then \(y_\beta = \pi_\beta(\bf{y}) \in V_\beta\), so \(y_\beta \in V_\beta \cap A_\beta \neq \varnothing\), and we conclude that \(x_\beta \in \cl{A_\beta}\).

Now there are some properties that don't hold for both topologies. You will see why we prefer the product topology.

\thm{19.6} For the \textbf{product space} \(\prod_{\alpha \in J} X_\alpha\), let \(f: A \ra \prod_{\alpha \in J} X_\alpha\), defined by
\[
    f(a) = \paren{f_\alpha\paren{a}}_{\alpha \in J}, \quad (a \in A)
\]
where \(f_\alpha : A \ra X_\alpha\). Then \(f\) is continuous if and only if \(f_\alpha\) is continuous for all \(\alpha \in J\).

\pf \note{\mimp} \(\pi_\beta\) is continuous. So \(f_\beta = \pi_\beta \circ f\) is continuous.\footnote{This implication also holds for the box topology.}

\note{\mimpd} If \(U_\beta\) is open in \(X_\beta\), note that \(\pi_\beta\inv(U_\beta)\) is a subbasis element. Then
\[
    f\inv\paren{\pi_\beta\inv\paren{U_\beta}} = (\pi_\beta \circ f)\inv(U_\beta) = f_\beta\inv(U_\beta)
\]
is open, so \(f\) is continuous.

Note that we had to use the fact that \(\pi_\beta\inv(U_\beta)\) is a subbasis element.

\ex. Let \(\R^\omega = \prod_{n \in \N}\R\). Consider the function \(f: \R \ra \R^\omega\) defined as \(f(t) = (t, t, \dots)\), \((t\in \R)\). If \(\R^\omega\) is given as a product topology, \(f\) is continuous by the above theorem. But \(f\) is not continuous, if \(\R^\omega\) is given by the box topology. Take \(B = (-1, 1) \times \paren{-\frac{1}{2}, \frac{1}{2}} \times \paren{-\frac{1}{3}, \frac{1}{3}} \times \cdots\). This is an open set in the box topology (but not in the product topology), and \(f\inv(B) = \{0\}\), which is not open.

\topic{The Metric Topology}

This is one of the most important topology among the topologies we learn.

\defn. \note{Metric} A \textbf{metric} on a set \(X\) is a function \(d: X\times X \ra \R\) having the following properties.
\begin{enumerate}
    \item For all \(x, y \in X\), \(d(x, y) \geq 0\) and equality holds when \(x = y\).
    \item For all \(x, y \in X\), \(d(x, y) = d(y, x)\).
    \item For all \(x, y, z \in X\), \(d(x, y) + d(y, z) \geq d(x, z)\).
\end{enumerate}

\defn. Given \(\epsilon > 0\),
\[
    B_d(x, \epsilon) = \{y \in X : d(x, y) < \epsilon\}
\]
is called the \textbf{\(\epsilon\)-ball centered at \(x\)}.

\defn. \note{Metric Topology} Let a metric \(d\) be given on \(X\). Then the topology generated by the basis
\[
    \mc{B} = \{B_d(x, \epsilon) : x\in X, \epsilon > 0\}
\]
is called the \textbf{metric topology} induced by \(d\).

We need to check that \(\mc{B}\) is indeed a basis. First of all, since if \(y \in B_d(x, \epsilon)\), then \(B_d(y, \delta) \subseteq B_d(x, \epsilon)\) for \(\delta = \epsilon - d(x, y) > 0\). The first condition for a basis is clear. So let \(y \in B_1\cap B_2\), with \(B_1 = B_d(x_1, \epsilon_1)\), \(B_2 = B_d(x_2, \epsilon_2)\). Then there exists \(\delta_1, \delta_2 > 0\) such that \(y \in B_d(y, \delta_1) \subseteq B_1\), \(y \in B_d(y, \delta_2) \subseteq B_2\). So setting \(B_3 = B_d(y, \min\{\delta_1, \delta_2\})\) gives \(y \in B_3 \subseteq B_1 \cap B_2\).

\rmk \(U\) is open in the metric topology \(\iff \forall y \in U\), \(\exists \delta > 0\) such that \(B_d(y, \delta) \subset U\).

\pf \note{\mimpd} Trivial. \note{\mimp} Use the fact that the balls form a basis.

\ex.
\begin{enumerate}
    \item Let \(X\) be a set. Define a metric by \(d(x, y) = 0\) if \(x = y\), and \(1\) otherwise. Then any one-point set \(\{x\}\) can be written as \(B_d\paren{x, \frac{1}{2}}\), so it is open (and also closed). We can see that this metric gives the discrete topology.
    \item The metric \(d(x, y) = \abs{x - y}\) will give the standard topology.
\end{enumerate}

\defn. A topological space \((X, \mc{T})\) is \textbf{metrizable} if there exists a metric \(d\) on \(X\) such that \(\mc{T}\) is equal to the metric topology induced by \(d\). We call \((X, d)\) a \textbf{metric space}.\footnote{In the later chapters we will learn Urysohn's metrization theorems, which is one of the important results, telling us about when a topological space is metrizable.}

\defn. Let \((X, d)\) be a metric space. For \(A \subseteq X\), \(A\) is \textbf{bounded} if there exists \(M \geq 0 \) such that \(d(x, y) \leq M\) for any \(x, y \in A\). If \(A\) is bounded, we define
\[
    \diam A = \sup\{d(x, y) : x, y \in A\}.
\]

The notion of boundedness is not a topological property.

\thm{20.1} Given a metric space \((X, d)\), define \(\bar{d}: X\times X \ra \R\) as \(\bar{d}(x, y) = \min\{d(x, y), 1\}\). Then \(\bar{d}\) is a metric that induces the same topology as \(d\).

\pf First, we show that \(\bar{d}\) is a metric. If \(d(x, y) \geq 1\) or \(d(y, z) \geq 1\), then \(\bar{d}(x, z) \leq 1 \leq \bar{d}(x, y) + \bar{d}(y, z)\). Otherwise, \(\bar{d} = d\), so the triangle inequality also holds. The other conditions are trivial.

Now, to show that \(\bar{d}\) induces the same topology, check that \(\{B_d(x, \epsilon) : x \in X, 0 < \epsilon < 1\}\) is a basis for the metric topology induced by \(d\). If we change \(d\) to \(\bar{d}\), the statement also holds.

\pagebreak
