\section*{April 25th, 2023}

\topic{Local Compactness}

\defn. \note{Local Compactness} A space \(X\) is \textbf{locally compact} at \(x\) if there exists a compact subspace \(C\) of \(X\) that contains a neighborhood of \(x\). A space \(X\) is locally compact if it is locally compact for every \(x \in X\).

\ex.
\begin{enumerate}
    \item \(\R\) is not compact, but it is locally compact. We can always take a closed interval containing a point. Similarly, \(\R^n\) is also locally compact.
    \item \(\R^\omega\) with the product topology is not locally compact. Suppose not, take a basis element \(B = (a_1, b_1) \times \cdots \times (a_n, b_n) \times \R \times \cdots\) containing \(x \in \R^\omega\). Then \(B\) is contained in a compact subspace. ... Also, \(\cl{B} = [a_1, b_1] \times \cdots \times [a_n, b_n] \times \R \times \cdots\) should be compact (subset of \(C\)) but it isn't.
    \item \(\Q\) is not locally compact. (Check!)
\end{enumerate}

\thm{29.1} Let \(X\) be a topological space. \(X\) is locally compact and Hausdorff if and only if there exists a topological space \(Y\) such that
\begin{enumerate}
    \item \(X\) is a subspace of \(Y\)
    \item \(Y \bs X\) consists of a single point.
    \item \(Y\) is a compact Hausdorff space.
\end{enumerate}
Moreover if \(Y, Y'\) are two spaces satisfying the three conditions, there exists homeomorphism from \(Y\) to \(Y'\) that equals the identity map on \(X\).

If we are given a \textit{locally compact} Hausdorff space, we can find a \textit{compact} Hausdorff space, by adding a single point. Consider \(S^1\) with the north pole removed. By stereographic projection, we have a homeomorphism with \(\R\). We can also see that \(S^n\) except the north pole is homeomorphic to \(\R^n\). Keep this example in mind.

\pf \note{Step 1} Existence of homeomorphism. (Uniqueness up to homeomorphism) \\
Suppose that \(Y, Y'\) are the topological space satisfying the conditions. Construct a homeomorphism \(h : Y \ra Y'\) by \(h(x) = x\) for \(x \in X\), \(h(p) = q\) where \(p \in Y \bs X\), \(q \in Y' \bs X\). Next we show that \(h\) is an open map. Let \(U\) be open in \(Y\).

If \(p \notin U\), \(h(U) = U\). \(U\) is open in \(X\), \(X\) is open in \(Y'\), so \(U\) is open in \(Y'\). If \(p \in U\), then \(Y \bs U \subset X\). \(Y \bs U\) is closed in \(Y\), so it is a compact subspace of \(Y\). Then it is also a compact subspace of \(X\), and also \(Y'\). So \(Y \bs U\) is closed in \(Y'\). (Hausdorff) Then \(h(U) = Y' \bs (Y \bs U)\) is open in \(Y'\). If we change the role of \(Y\) and \(Y'\), we can show that \(h\inv\) is also an open map. \(h\) is an homeomorphism.

\note{Step 2} \note{\mimp} Construction of \(Y\). \\
Let \(Y = X \cup \{\infty\}\). We give a topology on \(Y\). \(U \in \mc{T}_Y\) if either \(U \in \mc{T}_X\) (type 1) or \(U = Y \bs C\) where \(C\) is a compact subspace of \(X\) (type 2).\footnote{We want the interval that contains the punctured point.} We need to check that this is a topology.
\begin{itemize}
    \item \(\varnothing\), \(Y\) are indeed in \(\mc{T}_Y\).
    \item \(U_1\cap U_2\) is open. For \((Y \bs C_1) \cap (Y \bs C_2) = Y \bs (C_1 \cup C_2)\), and finite union of compact set is compact. Also, \(U_1 \cap (Y \bs C_1) = U_1 \cap (X \bs C_1)\), and \(X - C_1\) is open in \(X\).
    \item For arbitrary union, \(\bigcup U_\alpha\) is open, \(\bigcup (Y \bs C_\beta) = Y \bs \bigcap C_\beta\) is also open since \(\bigcap C_\beta\) is closed subset of a compact space \(C_\beta\). Next, \(\bigcup U_\alpha \cup \bigcup (Y - C_\beta) = Y \bs (\bigcap C_\beta \bs \bigcup U_\alpha)\), and \(\bigcap C_\beta \bs \bigcup U_\alpha\) is compact since it is a closed subspace of \(\bigcap C_\beta\).
\end{itemize}

Now \(Y\) is a topological space, next we show that \(X\) is a subspace of \(Y\). For open sets of the subspace topology on \(X\), if \(U \in \mc{T}_X\), \(U \cap X = U\). Also \((Y \bs C) \cap X = X \bs C\) is open in \(X\). The opposite direction is clear.

Next we show that \(Y\) is compact. Let \(\mc{A}\) be an open covering of \(Y\). \(\mc{A}\) should contain an open set of type 2, since open sets of type 1 cannot cover \(\{\infty\}\). Suppose that \(Y \bs C \in \mc{A}\) where \(C\) is compact. Then there exists a finite subcover of \(C\). So union the subcover with \(Y \bs C\) to get a finite subcover of \(Y\).

Finally we show that \(Y\) is Hausdorff. Let \(x, y \in Y\). If \(x, y \in X\), then there is nothing to check. Consider the case where \(x \in X\), \(y = \infty\). Since \(X\) is locally compact, there exists a compact subspace \(C\) of \(X\) containing a neighborhood \(U\) of \(x\). Then \(Y \bs C\) is a neighborhood of \(y\), and it is disjoint with \(U\).

\note{Step 3} \note{\mimpd} Construction of \(X\).
\(X\) is Hausdorff since it is a subspace of a Hausdorff space \(Y\). We show that \(X\) is locally compact. For \(x \in X\), there exists disjoint neighborhoods \(U, V\) of \(x\) and \(\infty\) in \(Y\), respectively. Then \(U \subset Y \bs V\) and \(Y \bs V\) is closed, so it is a compact subspace of \(Y\). \(Y \bs V\) is also compact in \(X\). \qed

\pagebreak

\rmk If \(X\) is already compact, \(\{\infty\}\) is open in \(Y\). Otherwise, \(\infty\) is a limit point of \(X\), so \(\cl{X} = Y\). We say that \(Y\) is a compactification of \(X\).

\defn. \note{Compactification} For a compact Hausdorff space \(Y\), if \(X\) is a proper subspace of \(Y\) and \(Y = \cl{X}\), \(Y\) is a \textbf{compactification} of \(X\). In particular, if \(Y \bs X\) is a singleton, then \(Y\) is the one-point compactification of \(X\).

\rmk The one-point compactification of \(\R^n\) is homeomorphic to \(S^n\).

You will see many times that compact Hausdorff spaces are very important. Given a space \(X\), if you are lucky to find a compactification \(Y\) of \(X\), properties of \(Y\) also apply to \(X\).

An open disk is homeomorphic to \(\R^2\). You can add the boundary \(S^1\) to obtain a compactification.

Locally compactness is not actually a local property.

\thm{29.2} Let \(X\) be a Hausdorff space. Then \(X\) is locally compact if and only if
\begin{center}
    \(\forall x\in X\), \(\forall\) neighborhood \(U\) of \(x\), \(\exists\)neighborhood \(V\) of \(x\) such that \(\cl{V}\) is compact and \(\cl{V} \subset U\).
\end{center}

\pf \note{\mimpd} Clear. \\
\note{\mimp} \(X\) is locally compact, so let \(Y\) be the one-point compactification of \(X\). Let \(C = Y \bs U\). Then \(C\) is a compact subspace of \(Y\), and it is closed. By \sref{Lemma 26.4}, there exists disjoint open sets \(V\) and \(W\) of \(Y\) such that \(x \in V\) and \(C \subset W\). Then \(V \subset Y \bs W \subset Y \bs C \subset U\). Take the closure of \(V\) in \(Y\). Then \(\cl{V} \subset \cl{Y \bs W} = Y \bs W\), so \(\cl{V} \subset U\). Also \(\cl{V}\) is a closed subset of a compact space \(Y\), so it is compact, and \(X\) is open in \(Y\) so \(\cl{V}\) is compact in \(X\). \qed

\cor{29.3} Given a locally compact Hausdorff space \(X\), let \(A \subset X\). If \(A\) is open (or closed) in \(X\) then \(A\) is locally compact.

\pf For the closed case, check by yourself. \\
Let \(A\) be open. By \sref{Theorem 29.2}, for \(x \in A\), there exists a neighborhood \(V\) of \(x\) in \(X\) such that \(\cl{V}\) is compact and \(\cl{V} \subset A\). Replace \(X\) by \(A\). Then we are done. \qed

\cor{29.4} \(X\) is locally compact Hausdorff if and only if \(X\) is homeomorphic to an open subspace of a compact Hausdorff space.

\pf \note{\mimpd} By \sref{Corollary 29.3}, subspace of a compact Hausdorff space is locally compact Hausdorff. \(X\) is homeomorphic to this subspace, so \(X\) is locally compact Hausdorff. \\
\note{\mimp} By one-point compactification, \sref{Theorem 29.1}. \qed

\pagebreak
