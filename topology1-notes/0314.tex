\section*{March 14th, 2023}

\thm{18.2} \note{Constructing Continuous Functions} Let \(X, Y, Z\) be topological spaces.
\begin{enumerate}
    \item If \(f: X \ra Y\) is a constant function, then \(f\) is continuous.
    \item If \(A\) is a subspace of \(X\), the inclusion \(j : A \ra X\) is continuous.
    \item If \(f : X \ra Y\), \(g : Y \ra Z\) are continuous, then \(g \circ f : X \ra Z\) is continuous.
    \item If \(f: X \ra Y\) is continuous and \(A\) is a subspace, \(f|_A: A \ra Y\) is continuous.
    \item If \(f : X \ra Y\) is continuous and \(f(X) \subseteq Z \subseteq Y\), then \(g : X \ra Z\) is given by restricting the range, is continuous. If \(Y \subseteq W\), then \(h : X \ra W\) given by expanding the range, is continuous.\footnote{The subsets are all subspaces!}
    \item \(X = \bigcup_{\alpha \in J} U_\alpha\), \(U_\alpha\) open in \(X\), then \(f : X \ra Y\) is continuous if and only if \(f|_{U_\alpha}\) is continuous \(\forall \alpha \in J\).
\end{enumerate}

\pf \\
\note{1} Inverse image is empty or \(X\), so \(f\) is continuous.

\note{2} Let \(U\) be an open set in \(X\). Then \(j\inv(X) = U \cap A\) is open since \(A\) is a subspace.

\note{3} For any open set \(U\) in \(Z\), \(f\inv\paren{g\inv(U)}\) is open.

\note{4} Write \(f|_A = f \circ j\), where \(j: A \ra X\) is the inclusion map. The result follows from (3).

\note{5} Let \(B\) be open in \(Z\). Then \(B = Z \cap U\) for some open set \(U\) in \(Y\). Since \(f\inv(U) = g\inv(B)\) and \(f\inv(U)\) is open in \(X\), \(g\inv(B)\) is also open in \(X\). As for \(h\), write \(h = j \circ f\), where \(j : Y \ra W\) is an inclusion.

\note{6} \note{\mimp} Trivial. \\
\note{\mimpd} Let \(V\) be open in \(Y\). Then we see that \(f\inv(V)\cap U_\alpha = \paren{f|_{U_\alpha}}\inv(V)\) is open in \(U_\alpha\). Since \(U_\alpha\) is open in \(X\), the set \(f\inv(V)\cap U_\alpha\) is open, and \(f\inv(V) = \bigcup_{\alpha \in J} \paren{f\inv(V) \cap U_\alpha}\) is open.

\ex. Consider
\begin{enumerate}
    \item Let \(h(x) = x\) if \(x \leq 0\), \(\frac{x}{2}\) if \(x \geq 0\).
    \item Let \(l(x) = x - 2\) if \(x < 0\), \(x + 2\) if \(x \geq 0\).
\end{enumerate}
Each piece is continuous. Then why is \(h\) continuous, but \(l\) isn't? Because for \(h\), the pieces agree on the overlapping parts of their domains!

\thm{18.3} \note{The Pasting Lemma} Let \(X = A \cup B\) where \(A, B\) are closed in \(X\). If \(f : A \ra Y\), \(g : B \ra Y\) are continuous, and \(f(x) = g(x)\) for all \(x \in A \cap B\), then \(h: X \ra Y\) defined by
\[
    h(x) = \begin{cases}
        f(x) & (x \in A) \\ g(x) & (x \in B)
    \end{cases}
\]
is continuous.

\pf Let \(C\) be closed in \(Y\). Then \(h\inv(C) = f\inv(C) \cup g\inv(C)\), which is closed in \(X\). This is because \(f\inv(C), g\inv(C)\) are closed in \(X\), since \(A, B\) are closed in \(X\).

\thm{18.4} \note{Maps into Products} Let \(f : A \ra X \times Y\), with \(f(a) = \paren{f_1(a), f_2(a)}\). Then \(f\) is continuous if and only if \(f_1 : A \ra X\), \(f_2 : A \ra Y\) are continuous.

\pf \note{\mimp} The projections \(\pi_1, \pi_2\) are continuous, so write \(f_i: \pi_i \circ f\).

\note{\mimpd} Let \(U \times V\) be a basis element of \(X \times Y\). Then \(f\inv(U \times V) = f_1\inv(U) \cap f_2\inv(V)\).

\topic{The Product Topology}

Let \(X\) be a set, and \(J\) be an index set.

\defn.
\begin{enumerate}
    \item We call \(\bf{x} = J \ra X\) a \textbf{\(J\)-tuple}. We write \(\bf{x}(\alpha) = x_\alpha\) or \(\bf{x} = (x_\alpha)_{\alpha \in J}\).
    \item \(X^J\) is the set of all \(J\)-tuples of elements of \(X\).
\end{enumerate}

\defn. Let \(\seq{A_\alpha}_{\alpha \in J}\) be an indexed family of sets, and \(X = \bigcup_{\alpha \in J} A_\alpha\). We define
\[
    \prod_{\alpha \in J} A_\alpha
\]
the set of all \(J\)-tuples \((x_\alpha)_{\alpha \in J}\) of elements of \(X\) such that \(x_\alpha \in A_\alpha\) for all \(\alpha \in J\), i.e. the set of all functions \(\bf{x}: J \ra \bigcup_{\alpha \in J} A_\alpha\) such that \(\bf{x}(\alpha) \in A_\alpha\) for all \(\alpha \in J\).

\defn. Let \(\seq{X_\alpha}_{\alpha \in J}\) be an indexed family of topological spaces. We define
\[
    \prod_{\alpha \in J} X_\alpha
\]
\begin{enumerate}
    \item The \textbf{product topology}, generated by the subbasis \(\mc{S} = \bigcup_{\beta \in J} \mc{S}_\beta\) where
          \[
              \mc{S}_\beta = \left\{ \pi_\beta\inv(U_\beta) : U_\beta \text{ is open in } X_\beta \right\}
          \]
          and \(\pi_\beta: \prod_{\alpha \in J} X_\alpha \ra X_\beta\).
    \item The \textbf{box topology}, generated by the basis \(\mc{B}\) where
          \[
              \mc{B} = \left\{\prod_{\alpha \in J} U_\alpha : U_\alpha \text{ is open in } X_\alpha\right\}.
          \]
\end{enumerate}

\rmk
\begin{enumerate}
    \item If \(J\) is finite, the product topology and the box topology are the same! But these are different when \(J\) is infinite.
    \item The product topology is a subset of the box topology, and the equality holds when \(J\) is finite.
    \item For the product topology, the basis \(\mc{B}\) generated by the subbasis \(\mc{S}\), is the finite intersections of elements of \(\mc{S}\). So
          \[
              \mc{B} = \pi_{\beta_1}\inv(U_{\beta_1}) \cap \cdots \cap \pi_{\beta_n}\inv(U_{\beta_n}) = \prod_{\alpha \in J} U_\alpha
          \]
          where \(U_\alpha\) is open in \(X_\alpha\), and \(U_\alpha = X_\alpha\) if \(\alpha \neq \beta_1, \dots, \beta_n\). But we have an arbitrary choice of open sets in the box topology.

          Consider \(\R^\omega = \R^{\Z^+}\) then \((-1, 1) \times \cdots\) is a basis element of the box topology, but not a basis element of the product topology. However, \((-1, 1) \times \cdots \times \R \times \cdots\) is a basis element of the product topology.
\end{enumerate}

\pagebreak
