%!TEX encoding = utf-8
\documentclass[12pt]{report}
\usepackage{kotex}
\usepackage{amsmath}
\usepackage{amssymb}
\usepackage{eucal}
\usepackage{geometry}
\geometry{
    top = 15mm,
    left = 15mm,
    right = 15mm,
    bottom = 20mm
}
\geometry{a4paper}

\pagenumbering{gobble}
\renewcommand{\baselinestretch}{1.3}

\newcommand{\ra}{\rightarrow}
\newcommand{\cl}[1]{\overline{#1}}
\newcommand{\inv}{^{-1}}
\newcommand{\R}{\mathbb{R}}
\newcommand{\mc}[1]{\mathcal{#1}}

\renewcommand{\epsilon}{\varepsilon}

\DeclareMathOperator{\Bd}{Bd}

\begin{document}
\begin{center}
    \textbf{\large Introduction to Topology I Final}\\
    June 13, 2023
\end{center}

\hrule
\begin{enumerate}
    \item (10 pts.) A space \(X\) is said to be \textit{countably compact} if every countable open covering of \(X\) contains a finite subcollection that covers \(X\). Show that for a metrizable space \(X\), countable compactness is equivalent to compactness.

    \item (10 pts.) Consider the following subspaces of the plane \(\R^2\):
          \begin{center}
              \(X = \{x \times y \mid 1 < x^2 + y^2 \leq 4\}\) \quad and \quad \(Y = \{x \times y \mid x^2 + y^2 \leq 1\}\).
          \end{center}
          Show that the one-point compactification of \(X\) is homeomorphic with \(Y\).

    \item (10 pts.) Let \(p : X \ra Y\) be a closed continuous surjective map such that \(p\inv(\{y\})\) is compact for each \(y \in Y\). Show that if \(X\) is second-countable, then so is \(Y\).

    \item (10 pts.) Show that every regular Lindelöf space is normal.

    \item (10 pts.) Let \(X\) be a compact Hausdorff space. Show that \(X\) is metrizable if and only if \(X\) has a countable basis.

    \item (10 pts.) Show that if \(Y\) is locally compact Hausdorff, then composition of maps
          \[
              \mc{C}(X, Y) \times \mc{C}(Y, Z) \ra \mc{C}(X, Z)
          \]
          is continuous, provided the compact-open topology is used throughout.

    \item (10 pts.) Let \(X\) be a complete metric space. Show that if \(Y\) is a \(G_\delta\) set in \(X\), then \(Y\) is a Baire space in the subspace topology.

    \item[Bonus] (\(5 + 5 = 10\) pts.) Let \((X, d)\) be a metric space. If \(A \subset X\) and \(\epsilon > 0\), let
        \[
            U(A, \epsilon) = \{x \mid d(x, A) < \epsilon\}
        \]
        be the \(\epsilon\)-neighborhood of \(A\). Let \(\mc{H}\) be the collection of all (nonempty) closed, bounded subsets of \(X\). If \(A, B \in \mc{H}\), define
        \[
            D(A, B) = \inf\left\{\epsilon \mid A \subset U(B, \epsilon) \text{ and } B \subset U(A, \epsilon)\right\}.
        \]
        \begin{enumerate}
            \item Show that \(D\) is a metric on \(\mc{H}\); it is called the \textit{Hausdorff metric}.
            \item Show that if \((X, d)\) is totally bounded, so is \((\mc{H}, D)\).

                      [\textit{Hint}: Given \(\epsilon\), choose \(\delta < \epsilon\) and let \(S\) be a finite subset of \(X\) such that the collection \(\{B_d(x, \delta) \mid x \in S\}\) covers \(X\). Let \(\mc{A}\) be the collectioin of all nonempty subsets of \(S\); show that \(\{B_D(A, \epsilon) \mid A \in \mc{A}\}\) covers \(\mc{H}\).]
        \end{enumerate}
\end{enumerate}

\end{document}
