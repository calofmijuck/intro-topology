%!TEX encoding = utf-8
\documentclass[12pt]{report}
\usepackage{kotex}
\usepackage{geometry}
\geometry{
    top = 20mm,
    left = 15mm,
    right = 20mm,
    bottom = 20mm
}
\geometry{a4paper}

\pagenumbering{gobble}
\renewcommand{\baselinestretch}{1.4}
\newcommand{\prob}[1]{\item[\large\textbf{\sffamily #1.}]}
\newcommand{\subprob}[1]{\item[\textbf{\sffamily (#1)}]}

\newcommand{\ds}{\displaystyle}

\newcommand{\mf}[1]{\mathfrak{#1}}
\newcommand{\mc}[1]{\mathcal{#1}}
\newcommand{\bb}[1]{\mathbb{#1}}
\renewcommand{\bf}[1]{\mathbf{#1}}

\newcommand{\inv}{^{-1}}
\newcommand{\adj}{\text{*}}
\newcommand{\bs}{-}

\newcommand{\norm}[1]{\left\lVert #1 \right\rVert}
\newcommand{\abs}[1]{\left| #1 \right|}
\newcommand{\paren}[1]{\left( #1 \right)}
\newcommand{\seq}[1]{\left\{ #1 \right\}}
\renewcommand{\span}[1]{\left\langle #1 \right\rangle}

\newcommand{\ra}{\rightarrow}
\newcommand{\imp}{\implies}
\newcommand{\mimp}{\(\implies\)}
\newcommand{\mimpd}{\(\impliedby\)}
\newcommand{\miff}{\!\!\(\iff\)}
\newcommand{\mast}{\(\ast\)}
\newcommand{\mstar}{\(\star\)}

\newcommand{\R}{\mathbb{R}}
\newcommand{\N}{\mathbb{N}}
\newcommand{\Z}{\mathbb{Z}}
\newcommand{\Q}{\mathbb{Q}}
\newcommand{\C}{\mathbb{C}}

\DeclareMathOperator{\inte}{Int}
\newcommand{\cl}[1]{\overline{#1}}
\DeclareMathOperator{\diam}{diam}

\renewcommand{\bar}[1]{\overline{#1}}

\let\oldexists\exists
\renewcommand{\exists}{\oldexists\,}

\newcommand{\defn}[1]{%
    \ifthenelse{\equal{#1}{.}}
    {\textbf{\sffamily Definition.}}%
    {\textbf{\sffamily Definition #1.}}%
}

\newcommand{\thm}[1]{%
    \ifthenelse{\equal{#1}{.}}
    {\textbf{\sffamily Theorem.}}%
    {\textbf{\sffamily Theorem #1.}}%
}

\newcommand{\prop}[1]{%
    \ifthenelse{\equal{#1}{.}}
    {\textbf{\sffamily Proposition.}}%
    {\textbf{\sffamily Proposition #1.}}%
}

\newcommand{\ex}[1]{%
    \ifthenelse{\equal{#1}{.}}
    {\textbf{\sffamily Example.}}%
    {\textbf{\sffamily Example #1.}}%
}

\newcommand{\prob}[1]{%
    \ifthenelse{\equal{#1}{.}}
    {\textbf{\sffamily Problem.}}%
    {\textbf{\sffamily Problem #1}}%
}

\newcommand{\lemma}[1]{%
    \ifthenelse{\equal{#1}{.}}
    {\textbf{\sffamily Lemma.}}%
    {\textbf{\sffamily Lemma #1.}}%
}

\newcommand{\cor}[1]{%
    \ifthenelse{\equal{#1}{.}}
    {\textbf{\sffamily Corollary.}}%
    {\textbf{\sffamily Corollary #1.}}%
}

\newcommand{\recall}{\textbf{\sffamily Recall.\;}}
\newcommand{\rmk}{\textbf{\sffamily Remark.\;}}
\newcommand{\pf}{\textbf{\sffamily Proof.\;}}
\newcommand{\question}{\textbf{\sffamily Question.\;}}
\newcommand{\notation}{\textbf{\sffamily Notation.\;}}

\newcommand{\claim}{\textbf{Claim}}

\newcommand{\note}[1]{({\sffamily #1})}
\newcommand{\sref}[1]{{\sffamily #1}}

\newcounter{topic}
\setcounter{topic}{11}
\newcommand{\topic}[1]{
    \addtocounter{topic}{1}
    \vspace{20pt}
    \textbf{\large \S\thetopic\,\,#1}
    \vspace{3pt}
    \hrule
    \vspace{10pt}
}


\newcommand{\B}{\mathcal{B}}
\newcommand{\T}{\mathcal{T}}

\begin{document}
\begin{center}
    \textbf{\Large Introduction to Topology 1 - HW \#1}\\
    \large 2017-18570 Sungchan Yi
\end{center}
\begin{enumerate}
    \prob{1}
    \begin{enumerate}
        \subprob{a} First we check that \(\B\) is a basis. For all \(x \in \R\) choose \(a < x < b\), \(a, b \in \Q\). Then \(x \in (a, b) \in \B\).

        Let \(B_1 = (a, b), B_2 = (c, d)\) where \(a, b, c, d\in \Q\). Then we see that the intersection \(B_1 \cap B_2\) is either an empty set or an interval with endpoints in \(\Q\). Therefore we can always find \(B_3 \subset B_1\cap B_2\) with \(B_3 \in \B\). \(\B\) is a basis.

        Next, we show that this basis generates the standard topology on \(\R\). Let \(\T'\) be the standard topology on \(\R\), with basis \(\B' = \{(\alpha, \beta) : \alpha, \beta \in \R\}\). Also let \(\T\) be the topology generated by \(\B\).

        For any \(x \in \R\), choose a basis element \((\alpha, \beta) \in \B'\) containing \(x\). Then there exists \(r, s \in \Q\) such that \(\alpha < r < x < s < \beta\). Then \(x \in (r, s) \subset (\alpha, \beta)\) and \((r, s) \in \B\). Therefore \(\T' \subset \T\).

        For \(x \in \R\), choose basis element \((a, b) \in \B\) containing \(x\). There exists \(\alpha, \beta \in \R\) such that \(a < \alpha < x < \beta < b\). Then \(x \in (\alpha, \beta) \subset (a, b)\) and \((\alpha, \beta) \in \B'\). So \(\T \subset \T'\).

        Therefore \(\T = \T'\).

        \subprob{b} First we check that \(\mc{C}\) is a basis. For \(x \in \R\), choose \(a \leq x < b\) where \(a, b \in \Q\). Then \(x \in [a, b) \in \mc{C}\).

        Let \(B_1 = [a, b), B_2 = [c, d)\) where \(a, b, c, d \in \Q\), and let \(x \in B_1 \cap B_2\). If \(B_1 \cap B_2\) is either \(B_1\) or \(B_2\), we are done. Otherwise, without loss of generality, let \(c < b\). Then \(x \in [b, c) \subset B_1\cap B_2\) and \([b, c) \in \mc{C}\). \(\mc{C}\) is a basis.

        Next, we show that the topology \(\T\) generated by \(\mc{C}\) is different from the lower limit topology \(\R_l\). We claim that \(\R_l\) is strictly finer than \(\T\).

        Take a basis element \([a, b) \in \mc{C}\). This element is also a basis element of \(\R_l\).

        On the other hand, take a basis element \([\alpha, \beta)\) in \(\R_l\) where \(\alpha \in \R \bs \Q\) and \(\beta \in \R\). There is no interval of the form \([r, s) \in \mc{C}\) that contains \(\alpha\) and is a subset of \([\alpha, \beta)\).

        Thus \(\T\) is different from \(\R_l\).
    \end{enumerate}

    \prob{2}
    \begin{enumerate}
        \subprob{a} We show that
        \[
            \T_3 \subsetneq \T_1 \subsetneq \T_2 \subsetneq \T_4, \quad \T_5 \subsetneq \T_1,
        \]
        and that \(\T_3\) and \(\T_5\) are not comparable. From this relation among the five topologies, we can determine all 20 cases.

        \medskip

        (\(\T_3 \subsetneq \T_1\)) Take any basis element \(B \in \T_3\). Then \(\R \bs B\) is finite. Let \(\R - B = \{a_1, \dots, a_n\}\). Without loss of generality, let \(a_1 < a_2 < \cdots < a_n\). For convenience we will write \(a_0 = -\infty, a_{n+1} = +\infty\). Now for any \(x \in B\), we can choose \(0 \leq i \leq n\) such that \(a_i < x < a_{i+1}\). Then take basis element of \(\T_1\) as \((a_i, a_{i+1}) \subset B\).

        Take a basis element \(B = (0, 1) \in \T_1\). For \(x \in B\), there is no subset \(B'\) of \((0, 1)\) containing \(x\), with \(\R \bs B'\) finite. This is because \(\R \bs (0, 1)\) is already an infinite set.

        \medskip

        (\(\T_1 \subsetneq \T_2\)) Take any basis element \((a, b) \in \T_1\). This element is also a basis element of \(\T_2\).        \smallskip

        Take \(B = (-1, 1) \bs K \in \T_2\). For \(0 \in B\), there is no open interval that contains \(0\) and lies in \(B\).

        \medskip

        (\(\T_2 \subsetneq \T_4\)) Take basis element \(B \in \T_2\).
        \begin{itemize}
            \item If \(B = (a, b)\), for \(x \in B\) choose \((a, x] \subset (a, b)\) as a basis element of \(\T_4\).
            \item If \(B = (a, b) \bs K\), let \(x \in B\). If \(x \neq 0\), we can always choose an interval containing \(x\), since if \(0 < x < 1\), \(\exists n\) such that \(\frac{1}{n+1} < x < \frac{1}{n}\). Then choose interval \(\left(\frac{1}{n+1}, \frac{1/n + x}{2}\right]\). If \(x > 1\), choose \((\max\{a, 1\}, x]\). If \(x \leq 0\), choose \((a, x]\). (We assume \(a < x \leq 0\) to make \(x \in B\) possible)
        \end{itemize}
        For each case, we can always choose a subset that is a basis element of \(\T_4\).

        However, take basis element \(B = (0, 1] \in \T_4\). For \(x = 1\), no set in \(\T_2\) can contain \(x = 1\) and lie in \(B\). If an open interval contains \(x = 1\), a part of it lies outside of \(B\), and basis elements of the form \((a, b) \bs K\) do not contain \(x = 1\) in the first place.

        \medskip

        (\(\T_5 \subsetneq \T_1\)) Take basis element \(B = (-\infty, a) \in \T_5\). For \(x \in B\), we can choose \((x - \delta, x + \delta) \in \T_1\) where \(\delta = \frac{a-x}{2}\). Then \(x \in (x - \delta, x + \delta) \subset B\).

        Take basis \(B = (0, 1) \in \T_1\). For \(x \in (0, 1)\) there is no set of the form \((-\infty, a)\) that is a subset of \(B\).

        \medskip

        (\(\T_3\) and \(\T_5\) are not comparable) Take any basis element \(B \in \T_3\). Then \(\R \bs B\) is finite. Let \(\R - B = \{a_1, \dots, a_n\}\). Without loss of generality, let \(a_1 < a_2 < \cdots < a_n\). Now consider \(x = \frac{a_1 + a_2}{2} \in B\). Any basis element \(B' \in \T_5\) that contains \(x\) should also contain \(a_1\), but \(a_1 \notin B\). So we cannot find a basis element \(B' \in \T_5\) such that \(x \in B' \subset B\).

        Take basis element \(B = (-\infty, a) \in \T_5\). For \(x \in B\), suppose that there exists \(B' \in \T_3\) such that \(x \in B' \subset B\). But \(\R \bs B \subset \R \bs B'\). Since \(\R \bs B\) is not finite, \(\R \bs B'\) cannot be finite. Therefore such \(B' \in \T_3\) does not exist.

        Therefore \(\T_3\) and \(\T_5\) are not comparable.

        \subprob{b} We use the fact that \(\cl{K} = K \cup K'\) and try to find \(K'\) for each of the topologies.

        \medskip

        (\(\T_1\)) For \(x \neq 0\), there exists neighborhoods \(U\) of \(x\) such that \(U \cap K\) is \(\varnothing\) or \(\{x\}\). When \(x \in \R \bs K\), \(U \cap K = \varnothing\), and when \(x \in K\), \(U \cap K = \{x\}\). If \(x = 0\), any neighborhoods of \(0\), say \((a, b)\) will intersect \(K\) at \(\frac{1}{n} < b\). Therefore \(K' = \{0\}\), and \(\cl{K} = K \cup \{0\}\).

        \medskip

        (\(\T_2\)) For \(x = \frac{1}{n}\), \(n \in \N\), choose open set \(U = (0, 2) \bs K\), then \(U \cap K = \varnothing\). For \(x \neq \frac{1}{n}\), we can choose \((x - \epsilon, x + \epsilon)\) with \(\epsilon > 0\) small enough so that \((x - \epsilon, x + \epsilon) \cap K = \varnothing\). Therefore \(K' = \varnothing\), and \(\cl{K} = K\).

        \medskip

        (\(\T_3\)) Suppose that some \(x \in \R\) is not a limit point. Then there exists an open set \(U \in \T_3\) such that \(U \cap \paren{K \bs \{x\}} = \varnothing\). Then \(\R \bs U\) is infinite, since \(\paren{K \bs \{x\}} \subset \R \bs U\) and \(K \bs \{x\}\) is infinite. So such \(U \in \T_3\) does not exist and we conclude that \(K' = \R\), and \(\cl{K} = \R\).

        \medskip

        (\(\T_4\)) Let \(x \neq 0\). We can choose small enough \(\epsilon > 0\) so that \((x - \epsilon, x + \epsilon] \cap \paren{K \bs \{x\}} = \varnothing\). If \(x = 0\), Then \((-1, 0] \cap \paren{K \bs \{x\}} = \varnothing\). Therefore \(K' = \varnothing\), and \(\cl{K} = \R\).

        \medskip

        (\(\T_5\)) Let \(x \geq 0\). Then neighborhoods of \(x\) have the form \((-\infty, a)\) for \(0 \leq x < a\). So \((-\infty, a) \cap \paren{K \bs \{x\}}\) cannot be empty since \(\exists n \in \N\) with \(\frac{1}{n} < x\) if \(x > 0\) and if \(x = 0\), there exists \(n\) with \(\frac{1}{n} < a\). If \(x < 0\), take \((-\infty, a)\) for \(x < a < 0\). Then \((-\infty, a) \cap \paren{K \bs \{x\}} = \varnothing\). Therefore \(K'\) is the set of non-negative reals. \(\cl{K} = K \cup K'\), so \(\cl{K}\) is the set of non-negative reals.

        \subprob{c} By Theorem 17.8, if \(X\) is Hausdorff, \(X\) also satisfies the \(T_1\) axiom. We will check whether \(X\) satisfies the \(T_1\) axiom only when necessary.

        Let \(x_1, x_2 \in \R\), and without loss of generality, assume \(x_1 < x_2\).

        \medskip

        (\(\T_1\)) Take \(\delta = \frac{x_2 - x_1}{3}\) then \((x_1 - \delta, x_1 + \delta) \cap (x_2 - \delta, x_2 + \delta) = \varnothing\). So \(\T_1\) is Hausdorff and also satisfies the \(T_1\) axiom.

        \medskip

        (\(\T_2\)) The same argument used for \(\T_1\) also holds. So \(\T_2\) is Hausdorff and also satisfies the \(T_1\) axiom.

        \medskip

        (\(\T_3\)) Let \(U_1, U_2\) be neighborhoods of \(x_1, x_2\), respectively. Then \(\R \bs U_1, \R \bs U_2\) are finite. Let
        \[
            \R \bs U_1 = \{a_1, \dots, a_n\}, \quad \R \bs U_2 = \{b_1, \dots, b_m\}.
        \]
        Without loss of generality, assume that \(a_1 < \cdots < a_n\), \(b_1 < \cdots < b_m\). Take any \(x > \max\{a_n, b_m\}\), then \(x \in U_1 \cap U_2\). So \(X\) is not Hausdorff.

        Every finite point set \(A\) is closed, since \(X \bs A\) is open. (\(X \bs \paren{X \bs A} = A\) is finite) So the \(T_1\) axiom holds.

        \medskip

        (\(\T_4\)) Take \(\delta = \frac{x_2 - x_1}{3}\) then \((x_1 - \delta, x_1 + \delta] \cap (x_2 - \delta, x_2 + \delta] = \varnothing\). So \(\T_4\) is Hausdorff and also satisfies the \(T_1\) axiom.

        \medskip

        (\(\T_5\)) We show that finite point sets are not closed. Suppose that \(\{x_0\}\) is closed. Then \(\R \bs \{x_0\}\) is open, so for all \(x \in \R \bs \{x_0\}\), there should be a basis element \((-\infty, a) \in \T_5\) such that \(x \in (-\infty, a) \subset \R \bs \{x_0\}\). But if we choose \(x > x_0\), \(x_0 \in (-\infty, a)\) but \(x_0 \notin \R\bs \{x_0\}\). So \(\{x_0\}\) is not closed. \(\T_5\) does not satisfy the \(T_1\) axiom, and is not Hausdorff either.
    \end{enumerate}

    \prob{3} Consider a basis element \(B \times C\) of \(X \times Y\), where \(B\) is open in \(X\) and \(C\) is open in \(Y\). We know that
    \[
        \pi_1(B\times C) = B, \quad \pi_2(B\times C) = C,
    \]
    so \(\pi_1, \pi_2\) map basis elements to open sets.

    Now for arbitrary opens sets of \(X \times Y\), we use the fact that open sets can be represented as a union of basis elements. Since union of open sets are also open, it remains to show that for any function \(f : X \ra Y\) and \(A_\alpha \subset X\),
    \[\tag{\(\ast\)}
        f\paren{\bigcup_{\alpha \in J} A_\alpha} = \bigcup_{\alpha \in J} f\paren{A_\alpha}.
    \]
    Then the image of any open set in \(X \times Y\) by \(\pi_1, \pi_2\) will be a union of open sets, and we can conclude that \(\pi_1, \pi_2\) are open maps.

    Proof of \((\ast)\):
    \[
        \begin{aligned}
            y \in f\paren{\bigcup_{\alpha \in J} A_\alpha} & \iff \exists x \in \bigcup_{\alpha \in J} A_\alpha,\, y = f(x) \\
                                                           & \iff \exists \beta\in J, \exists x \in A_\beta, \, y = f(x)    \\
                                                           & \iff \exists \beta \in J,\,  y \in f(A_\beta)                  \\
                                                           & \iff y \in \bigcup_{\alpha \in J} f(A_\alpha).
        \end{aligned}
    \]

    \prob{4} A typical basis element of the topology inherited by \(L\) will be given as the intersection between a basis element of \(\R_l \times \R\) (or \(\R_l \times \R_l\)) and \(L\). Cases where \([a, b) \times (c, d) \cap L = \varnothing\) have been ignored.

    \bigskip

    For \(\R_l \times \R\), basis elements have the form \([a, b) \times (c, d)\) (\(a, b, c, d \in \R\)), which is a box with only the left edge included. So \([a, b) \times (c, d) \cap L\) will be a line segment, and it may or may not contain endpoints depending on the slope of \(L\).
    \begin{itemize}
        \item If \(L\) is parallel to the \(x\)-axis, the intersection \([a, b) \times (c, d) \cap L\) will be half-open intervals of the form \([\alpha, \beta)\), \(\alpha, \beta \in L\). So this is the topology of \(\R_l\).
        \item If \(L\) is parallel to the \(y\)-axis, the intersection \([a, b) \times (c, d) \cap L\) will be open intervals of the form \((\alpha, \beta)\), \(\alpha, \beta \in L\). So this is the topology of \(\R\).
        \item For the cases when \(L\) is not parallel to any of the axes, the intersection \([a, b) \times (c, d) \cap L\) will be of the from \([\alpha, \beta)\) if \(L\) cross the left edge of the box. If \(L\) cross the top/bottom edge of the box, \([a, b) \times (c, d) \cap L\) will be of the form \((\alpha, \beta)\). (\(\alpha, \beta \in L\)) So this is the topology of \(\R_l\). (\((\alpha, \beta)\) is also open in \(\R_l\))
    \end{itemize}

    For \(\R_l \times \R_l\), basis elements have the form \([a, b) \times [c, d)\) (\(a, b, c, d \in \R\)), which is a box with left and bottom edge included. So \([a, b) \times [c, d) \cap L\) will also be a line segment, and it might contain endpoints depending on the slope of \(L\).
    \begin{itemize}
        \item If \(L\) is parallel to the axes, \([a, b) \times [c, d) \cap L\) will have the form \([\alpha, \beta)\), \(\alpha, \beta \in L\). This is because \(L\) must pass through the bottom or left edge of the box. So this is the topology of \(\R_l\).
        \item If \(L\) is not parallel to any of the axes, we consider two cases.
              \begin{enumerate}
                  \item If the slope of \(L\) is positive, \([a, b) \times [c, d) \cap L\) will have the form \([\alpha, \beta)\) if \(L\) crosses the left or bottom edge of the box. So this is the topology of \(\R_l\).
                  \item If the slope of \(L\) is negative, \(L\) can only intersect only at the bottom left vertex of the box, which makes \([a, b) \times [c, d) \cap L\) a one-point set. So this is the discrete topology on \(\R\).
              \end{enumerate}
    \end{itemize}

    \prob{5} Assume that \(X\) is Hausdorff, and let \(\B\) be a basis of \(X\). We show that \(\Delta = \cl{\Delta}\). \(\Delta \subset \cl{\Delta}\) is trivial by definition.

    \bigskip

    Let \(a \in \cl{\Delta}\). Then for any open set \(B\) containing \(a\), \(B \cap \Delta \neq \varnothing\). Let \(a = x \times y \in X\times X\) and assume that \(x \neq y\). Since \(X\) is Hausdorff, there exist respective neighborhoods \(U_1, U_2\) of \(x, y\), where \(U_1 \cap U_2 = \varnothing\).

    Note that \(U_1 \times U_2\) is an open set in \(X \times X\) which contains \(a\). So it should be that \((U_1\times U_2) \cap \Delta \neq \varnothing\). But because \(U_1 \cap U_2 = \varnothing\), there is no point of the form \(x \times x\) in \(U_1 \cap U_2\). Thus \((U_1 \times U_2) \cap \Delta = \varnothing\). Contradiction, so \(x = y\) should hold, and then \(a \in \Delta\).

    \bigskip

    Suppose that \(\Delta\) is closed in \(X \times X\). Then \(\Delta = \cl{\Delta}\). Assume that \(X\) is not Hausdorff. Consider \(x, y \in X\), with \(x \neq y\). Then we claim that \(x \times y \in \cl{\Delta} = \Delta\), arriving at a contradiction.

    Take any basis element \(U_1 \times U_2\) containing \(x \times y\). Then \(U_1, U_2\) are neighborhoods of \(x, y\) respectively. Then by assumption that \(X\) is not Hausdorff, \(U_1 \cap U_2 \neq \varnothing\). Choose \(z \in U_1 \cap U_2\). Then \(z\times z \in U_1 \times U_2 \cap \Delta\), so \((U_1 \times U_2) \Delta \neq \varnothing\). Since the choice of \(U_1 \times U_2\) was arbitrary, \(x\times y \in \cl{\Delta} = \Delta\), which is a contradiction. Thus \(X\) is Hausdorff.

    \prob{6}
    \begin{enumerate}
        \subprob{a} \(\inte A \subset \cl{A}\), so \(\inte A \cap \bd A = \inte \cap \cl{A} \cap \cl{X - A} = \inte A \cap \cl{X - A}\). Let \(x \in \inte A\). Then there exists an open set \(U\) containing \(x\) and \(x \in U \subset A\). Since \(U \subset A\), \(U \cap (X - A) = \varnothing\), so \(x \notin \cl{X - A}\). Thus \(\inte A \cap \bd A = \varnothing\).

        \bigskip

        To show that \(\cl{A} = \inte A \cup \bd A\), first let \(x \in \cl{A}\).
        \begin{itemize}
            \item If \(x \in \cl{X - A}\), \(x \in \cl{A} \cap \cl{X - A}\), so \(x \in \bd A\).
            \item If \(x \notin \cl{X - A}\), there exists open set \(U\) containing \(x\) such that \((X - A) \cap U = \varnothing\). Then \(U \subset A\), so \(x \in \inte A\).
        \end{itemize}
        Therefore \(x \in \inte A \cup \bd A\), \(\cl{A} \subset \inte A \cup \bd A\).

        Also, since \(\inte A \subset \cl{A}\), \(\bd A \subset \cl{A}\) by definition, so \(\inte A \cup \bd A \subset \cl{A}\).

        \bigskip

        \(\therefore \cl{A} = \inte A \cup \bd A\).

        \subprob{b} We use the result from (a).
        \[
            \begin{aligned}
                \bd A = \varnothing & \iff \cl{A} = \inte A                                                       \\
                                    & \iff \inte A = A = \cl{A} \quad (\because \inte A \subset A \subset \cl{A}) \\
                                    & \iff A \text{ is both open and closed}.
            \end{aligned}
        \]

        \subprob{c} Since \(U, \bd U \subset \cl{U}\),
        \[
            \begin{aligned}
                U \text{ is open} & \iff \inte U = U                                       \\
                                  & \iff \cl{U} \bs \bd U  = U \quad (\because \text{(a)}) \\
                                  & \iff \bd U = \cl{U} \bs U.
            \end{aligned}
        \]

        \subprob{d} No. Consider \(U = (-1, 0) \cup (0, 1) \subset \R\). Then \(U\) is open, but \(\inte \cl{U} = \inte([-1, 1]) = (-1, 1) \neq U\).
    \end{enumerate}

    \prob{7}
    \begin{enumerate}
        \subprob{a} We show that
        \[
            X \bs \{x \in X : f(x) \leq g(x) \} = \{x \in X: f(x) > g(x)\}
        \]
        is open. Write
        \[
            \{x : f(x) > g(x)\} = \bigcup_{r \in Y} \{x : f(x) > r\} \cap \{x : g(x) < r\}
        \]
        Since \(f, g\) are continuous, for \(r \in Y\),
        \[
            \begin{aligned}
                F_r = \{x : f(x) > r\} = f\inv\paren{(r, \infty)}, \\
                G_r = \{x : g(x) < r\} = g\inv\paren{(-\infty, r)},
            \end{aligned}
        \]
        and \((r, \infty), (-\infty, r)\) are open, \(F_r, G_r\) are open sets. So \(F_r \cap G_r\) is open, thus
        \[
            \{x : f(x) > g(x)\} = \bigcup_{r \in Y} F_r \cap G_r
        \]
        is open, so \(\{x \in X : f(x) \leq g(x)\}\) is closed.

        \subprob{b} Let
        \[
            A = \{x \in X : f(x) \leq g(x)\}, \quad B = \{x \in X : f(x) \geq g(x)\}.
        \]
        \(A\) is closed by {\sf (a)}, and that \(B\) is closed can be proved similarly. Then \(X = A \cup B\), and \(f(x) = g(x)\) for every \(x \in A \cap B\). By the pasting lemma,
        \[
            h(x) = \begin{cases}
                f(x) & (x\in A) \\ g(x) & (x \in B)
            \end{cases}
        \]
        is continuous, and in fact, \(h\) is exactly equal to \(h(x) = \min\{f(x), g(x)\}\).
    \end{enumerate}

    \prob{8}
    \begin{enumerate}
        \subprob{a} For closed set \(V \subset Y\),
        \[
            f\inv(V) \cap A_\alpha = (f \mid_{A_\alpha})\inv(V),
        \]
        since
        \[
            \begin{aligned}
                x \in (f \mid_{A_\alpha})\inv(V) & \iff (f \mid_{A_\alpha})(x) \in V  \\
                                                 & \iff x \in A_\alpha, f(x) \in V    \\
                                                 & \iff x \in f\inv(V) \cap A_\alpha.
            \end{aligned}
        \]
        Since \(f \mid_{A_\alpha}\) is continuous, \((f \mid_{A_\alpha})\inv(V)\) is closed in \(A_\alpha\). \(A_\alpha\) is closed in \(X\), so \((f \mid_{A_\alpha})\inv(V)\) is closed in \(X\).
        \[
            f\inv(V) = \bigcup_{\alpha = 1}^n \paren{f\inv(V) \cap A_\alpha} = \bigcup_{\alpha = 1}^n (f \mid_{A_\alpha})\inv(V)
        \]
        is closed, since it is a finite intersection of closed sets. Thus \(f\) is continuous.

        \subprob{b} Consider \(f: [0, 1] \ra [0, 1]\), defined as the following:
        \[
            f(x) = \begin{cases}
                x & (0 \leq x < 1) \\ 0 & (x = 1)
            \end{cases}
        \]
        Let \(A_0 = \{1\}\), \(A_n = \left[0, 1 - \frac{1}{n}\right]\). Then \(\seq{A_n}_{n\geq 0}\) is a countable collection of closed sets and
        \[
            X = \bigcup_{n=0}^\infty A_n = [0, 1].
        \]
        \(f \mid_{A_0}\) is a constant function, and \(f \mid_{A_n}\) (\(n \geq 1\)) is the identity function, so \(f \mid_{A_n}\) is continuous.

        \bigskip

        But \(f\) is not continuous, since for open set \(\left[0, \frac{1}{2}\right) \subset Y\),
        \[
            f\inv\paren{\left[0, \frac{1}{2}\right)} = \left[0, \frac{1}{2}\right) \cup \{1\},
        \]
        which is not open in \([0, 1]\). Its complement \(\left[\frac{1}{2}, 1\right)\) is neither open nor closed.

        \subprob{c} For a closed subset \(V\) of \(Y\), we want to show that \(f\inv(V)\) is closed. We will show this by proving that \(\cl{f\inv(V)} \subset f\inv(V)\).

        Let \(x \in \cl{f\inv(V)}\). Then for some neighborhood \(U_x\) of \(x\), \(U_x \cap f\inv(V) \neq \varnothing\), and by the locally finite condition, \(U_x\) intersects \(A_{\alpha_1}, \dots, A_{\alpha_n}\). Since \(U_x \cap A_\alpha = \varnothing\) for \(\alpha \neq \alpha_1, \dots, \alpha_n\),
        \[
            U_x = \bigcup_{\alpha \in J} (U_x\cap A_\alpha) = \bigcup_{i=1}^n (U_x \cap A_{\alpha_i}).
        \]

        Using the fact that \(\bigcup_{i=1}^n (f \mid_{A_{\alpha_i}})\inv(V)\) is a closed set, we have
        \[
            \cl{\bigcup_{i=1}^n (f \mid_{A_{\alpha_i}})\inv(V)} = \bigcup_{i=1}^n (f \mid_{A_{\alpha_i}})\inv(V) \subset f\inv(V).
        \]
        So showing that \(x \in \cl{\bigcup_{i=1}^n (f \mid_{A_{\alpha_i}})\inv(V)}\) will be enough to prove that \(x \in f\inv(V)\).

        \bigskip

        Take any neighborhood \(W\) of \(x\). Then
        \[
            \begin{aligned}
                W \cap U_x \cap f\inv(V) & = W \cap \bigcup_{i=1}^n \paren{U_x \cap A_{\alpha_i} \cap f\inv(V)}          \\
                                         & = W \cap \bigcup_{i=1}^n \paren{U_x \cap \paren{f\mid_{A_{\alpha_i}}}\inv(V)} \\
                                         & \subset W \cap \bigcup_{i=1}^n \paren{f \mid_{A_{\alpha_i}}}\inv(V).
            \end{aligned}
        \]
        Noticing the fact that \(W \cap U_x\) is a neighborhood of \(x\), enables us to use that \(W \cap U_x \cap f\inv(V) \neq \varnothing\) (\(x \in \cl{f\inv(V)}\)) So for any neighborhood \(W\) of \(x\),
        \[
            W \cap \bigcup_{i=1}^n \paren{f \mid_{A_{\alpha_i}}}\inv(V) \neq \varnothing,
        \]
        Thus
        \[
            \begin{aligned}
                x \in \cl{\bigcup_{i=1}^n (f \mid_{A_{\alpha_i}})\inv(V)} = \bigcup_{i=1}^n (f \mid_{A_{\alpha_i}})\inv(V) \subset f\inv(V),
            \end{aligned}
        \]
        proving that \(\cl{f\inv(V)} \subset f\inv(V)\). \(f\) is continuous.
    \end{enumerate}

    \prob{9} \(\bf{x} \in \cl{\R^\infty}\) if and only if for all open neighborhoods \(U\) containing \(\bf{x}\), \(U \cap \R^\infty \neq \varnothing\).

    Let \(\bf{x} = (x_1, x_2, \dots) \in \R^\infty\), and suppose that there exists \(N \in \N\) such that \(x_n = 0\) for all \(n \geq N\).

    \bigskip

    For the box topology, let \(\bf{y} = (y_1, y_2, \dots) \notin \R^\infty\). Then there are infinitely many \(y_n \neq 0\). We now construct an open set \(U\) containing \(\bf{y}\) as the following.
    \[
        U = \prod_{n=1}^\infty I_n,
    \]
    where \(I_n = (-\epsilon, \epsilon)\) for some \(0 < \epsilon < 1\) if \(y_n = 0\), and \(I_n = (y_n - \delta, y_n + \delta)\) for some \(\delta = \abs{y_n} / 2\) if \(y_n \neq 0\). Notice that if \(y_n \neq 0\), \(0 \notin I_n\). Therefore, any element of \(U\) has infinitely many non-zeros, so \(U\) cannot intersect \(\R^\infty\) and \(\bf{y} \notin \cl{\R^\infty}\). If \(\bf{y} \in \R^\infty\), it is automatically an element of \(\cl{\R^\infty}\).

    Thus \(\cl{\R^\infty} = \R^\infty\), in the box topology.

    \bigskip

    We start from the same argument as the box topology. For the product topology, let \(\bf{y} = (y_1, y_2, \dots) \notin \R^\infty\). Then there are infinitely many \(y_n \neq 0\). We now construct an open set \(U\) containing \(\bf{y}\) as the following.
    \[
        U = \prod_{n=1}^\infty I_n,
    \]
    where \(I_n = (-\epsilon, \epsilon)\) for some \(0 < \epsilon < 1\) if \(y_n = 0\), and \(I_n = (y_n - \delta, y_n + \delta)\) for some \(\delta = \abs{y_n} / 2\) if \(y_n \neq 0\). But the difference here is that this can only be done up to finitely many \(n\), and for the rest of \(n\), \(I_n = \R\). Indeed, \(\bf{y}\) is still an element of \(U\).

    Suppose that \(I_n = \R\) for all \(n \geq M\), where \(M \in \N\) is a large enough. Then \(U\) will intersect with \(\R^\infty\) because for \(n \geq M\), \(0 \in I_n\). This is part of a sequence that is eventually zero, so \(\bf{y} \in \cl{\R^\infty}\). If \(\bf{y} \in \R^\infty\), it is automatically an element of \(\cl{\R^\infty}\).

    Thus \(\cl{\R^\infty} = \R^\omega\), in the product topology.
\end{enumerate}
\end{document}
